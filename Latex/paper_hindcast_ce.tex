\documentclass[preprint,12pt,authoryear]{elsarticle}

\usepackage{hyperref}
\usepackage{natbib}
\usepackage{graphicx}
\usepackage{ifthen}
\usepackage{tabularx}
\usepackage{lineno}
\usepackage{multirow}
\usepackage{longtable}
%\linenumbers*[1]

% add references to research questions in margin
\newcommand{\mrq}[2][]{}
\newcommand{\rrq}[3][]{}
\newboolean{ispaper}
\setboolean{ispaper}{true}

\def\appendixname{}

\journal{Coastal Engineering}

\begin{document}

\begin{frontmatter}

  \title{Simulating Spatiotemporal Aeolian Sediment Supply at a Mega Nourishment}

  \author[deltares]{Bas Hoonhout\corref{ca}}
  \ead{bas.hoonhout@deltares.nl}
  \cortext[ca]{Corresponding author}

  \author[tudelft]{Sierd de Vries}

  \address[deltares]{Deltares, Department of Hydraulic Engineering,
    Boussinesqweg 1, 2629HV Delft, The Netherlands.}

  \address[tudelft]{Delft University of Technology, Faculty of Civil
    Engineering and Geosciences, Department of Hydraulic Engineering,
    Stevinweg 1, 2628CN Delft, The Netherlands.}

  \begin{abstract}
    Mega nourishments are a novel approach to stimulating coastal
    safety and resilience. Mega nourishments are intended to spread
    along the coast on a decadal time scale by natural sediment
    transport processes with a minimum of intrusion into the natural
    coastal system. The supratidal morphodynamic behaviour of mega
    nourishments is not well understood due to complexities introduced
    by limitations in sediment availability to aeolian sediment
    transport. Consequently, the effectiveness of mega nourishments to
    stimulate coastal safety and to influence coastal landscape and
    habitat development remains unknown.

    In this paper we present a detailed 4-year hindcast of the
    morphological development of the Sand Motor mega nourishment in
    The Netherlands. We use the aeolian sediment transport and
    availability model \textsc{AeoLiS} that focuses specifically on
    the simulation of spatiotemporal variations in sediment
    availability. The model includes the recurrence relation between
    sediment availability and aeolian sediment transport through
    self-grading and beach armoring.

    We show that the model is able to reproduce multi-annual aeolian
    sediment transport rates in the Sand Motor domain in the four
    years after its construction. The RMSE is $3 \cdot 10^4 ~
    \mathrm{m^3}$ (7\% of the total sediment accumulation) and
    $\mathrm{R^2}$ is 0.93 when comparing timeseries of total sediment
    accumulation in the dunes, dune lake and lagoon. The combination
    of spatial and temporal variations in aeolian sediment
    availability, due to the combined influence of soil moisture,
    sediment sorting and beach armoring, is essential for an accurate
    estimate of the total sedimentation volume. The simulated feedback
    between aeolian sediment availability and transport is required
    for accurately describing compartmentalization of the beach and
    locating the aeolian sediment source areas in the Sand Motor
    domain.
    
  \end{abstract}

  \begin{keyword}
    aeolian sediment transport; aeolian sediment supply; beach
    armoring; mega nourishment; Sand Motor; numerical model; aeolis
  \end{keyword}

\end{frontmatter}

\section{Introduction}

% context
Mega nourishments are a novel approach to stimulate coastal safety and
resilience \citep{Stive2013}. By concentrating coastal interventions
in both time and space, mega nourishments are believed to strengthen
the natural coastal system and provide a cost-effective solution to
coastal hazards with a minimum of intrusion into the natural coastal
system. Mega nourishments are intended to spread along the coast on a
decadal time scale by natural sediment transport processes, while
stimulating natural coastal landscape and habitat development. Mega
nourishment are flexible enough to cope with uncertainties associated
with climate change. The Sand Motor in The Netherlands is the first
implementation of a mega nourishment worldwide.

Past research at the Sand Motor revealed particular morphodynamic
behaviour associated with mega nourishments that is not well
understood \citep{deSchipper2016, Huisman2016,
  Radermacher2017}. \citet{Hoonhout2017a} showed that the supratidal
morphodynamic behavior of the Sand Motor mega nourishment is highly
compartmentalized. The compartmentalization results in dune growth
rates that are lower than along the adjacent coasts with more regular
beaches. Despite beach widths up to 1 kilometer, aeolian sediment
transport rates remain modest due to limitations in sediment
availability.


%Hence, limitations in sediment availability also complicate
%the performance assessment of mega nourishments as measure to
%stimulate coastal safety and coastal landscape and habitat
%development. For unbiased assessment of these novel approaches to
%coastal safety and resilience with respect to alternative measures,
%quantification of the aeolian sediment flux in complex coastal
%configurations and availability-limited conditions is paramount.

%\citet{Hoonhout2017b} showed that
%the typical supratidal evolution of the Sand Motor is closely related
%to the nearshore spreading \citep{deSchipper2016} and sorting
%\citep{Huisman2016} of the nourishment sand, and the tidal flow
%dynamics around the nourishment \citep{Radermacher2017}.

% \citep{deSchipper2016, Huisman2016, Radermacher2017}

%Mega nourishments
%2017 paper
%Large fetches, no saturation

% what is done
Limitations in sediment availability complicate estimations of
sediment fluxes based on aeolian sediment transport models
\citep[e.g.][]{Jackson1999, Lynch2008, DavidsonArnott2009,
  Aagaard2014}. Consequently, aeolian sediment transport models
systematically overestimate the actual aeolian sediment flux
\citep{Sherman1998, Sherman2012}. Limitations in sediment availability
are traditionally incorporated in formulatios for equilibrium or
saturated sediment transport through a shear velocity threshold
\citep[e.g.][]{Howard1977, Dyer1986, Belly1964, Johnson1965,
  Hotta1984, Nickling1981, Arens1996, King2005}. But sediment
availability is governed by a variety of environmental factors and
inherently varies both in time and space. To incorporate the
spatiotemporal variations in sediment availability, various conceptual
frameworks have been developed.

\citet{Bauer2002} introduced the concept of critical fetch to account
for limitations in fetch and sediment availability and supply in
coastal sediment transport estimates. \citet{deVries2014b} used an
explicit source term in a one-dimensional advection formulation to
account for spatial variations in sediment
availability. \citet{Keijsers2016} introduced the behavioral DUBEVEG
model, as extension of the DECAL algorithm \citep{Baas2002}, that uses
probabilities to account for spatiotemporal differences in beach
erosion, dune development and vegetation growth. Although conceptually
useful, these concepts have limited predictive capabilities as the
critical fetch in \citet{Bauer2002}, the explicit source term in
\citet{deVries2014b} and the probabilities in the DUBEVEG model are
typically unknown a-priori. Therefore, various process-based
frameworks have been developed to simulate the spatiotemporal
variation in sediment availability. The simulated sediment
availability can then be fed to an aeolian sediment transport model to
obtain aeolian sediment transport fluxes for availability-limited
coastal systems.

\citet{vanDijk1999} and \citet{vanBoxtel1999} introduced an extensive
numerical model that simulates airflow over a given topography and
computes the spatiotemporal variation in aeolian sediment transport
including various limitations in sediment availability, like the
effect of precipitation and vegetation. Their model did not allow for
simulation of the limitations in sediment availability itself and is
computationally intensive due to the flow solver. \citet{Kroy2002}
introduced a more lightweight flow solver based on the model of
\citet{Weng1991}. \citet{Duran2013} extended this model with a
vegetation growth model and a water line to simulate the development
of coastal dunes. However, their model is focused on more traditional
coastlines as simulation of sediment availability is included only
through vegetation. \citet{Hoonhout2016} introduced the
\textsc{AeoLiS} model that focuses specifically on the simulation of
spatiotemporal variations in sediment availability, including the
recurrence relation between sediment availability and aeolian sediment
transport through self-grading and beach armoring. The model can be
used to obtain a lightweight, but versatile aeolian sediment transport
model that is suitable for availability-limited coastal environments.

%Equilibrium
%Air flow (Sauermann, Duran and Moore, Van Dijk, Van Boxtel)
%Grain trajectories (Andreotti)
%Thresholds (many)
%Fetch-limited (Bauer, Davidson-Arnott)
%Explicit availability (de Vries, Hoonhout)
%Behavioral (Baas and Nield DECAL, Baas 2002, Keijsers De Groot DUBEVEG=DECAL+VEG+MARINE?)
%
%a group of models can be identified wherein the primary objective is to
%reproduce, as far as possible, the complexities of the major controls on sediment
%erosion, transport and deposition and to enable real world prediction (e.g., van Dijk
%et al., 1999; Roelvink et al., 2009; Hounhout and de Vries, 2016; Berard et al., 2017).
%The primary aim of another group of morphological models is to isolate the effects of
%one or more key variables using a number of simplifying assumptions (e.g., Andreotti
%et al, 2010; Baas and Nield, 2007; Durán and Moore, 2013; Keijsers et al., 2016)
%
%Dijk en Boxtel: slow, not too good, lack supply, but similar intent

% what is needed
Practical and versatile aeolian sediment transport models with
predictive skill are a prerequisite for design of mega
nourishments. Optimization of designs on the effectiveness to increase
coastal safety and resilience, while stimulating natural coastal
landscape and habitat development, requires spatiotemporal
differentiation of aeolian sediment transport and availability. To
provide insight in the predictive skill of models for the long-term
development of the complex coastal environments that mega nourishment
typically are, long-term validation is required.

% what did we do
In this paper we present a detailed 4-year hindcast of the
morphological development of the Sand Motor mega nourishment in The
Netherlands using the aeolian sediment transport and availability
model \textsc{AeoLiS} \citep{Hoonhout2016}. We show that
spatiotemporal variations in sediment availability cause
compartmentalization of the mega nourishment and increase its lifetime
significantly. We also show that both spatial and temporal variability
in sediment availability are key to explain the long-term and seasonal
morphodynamic behaviour of the nourishment.

%% context
%In availability-limited coastal systems, the aeolian sediment
%transport rate is governed by the sediment availability rather than
%the wind transport capacity. Aeolian sediment transport models
%typically incorporate the sediment availability through the shear
%velocity threshold. However, the determination of appropriate
%threshold values in practice appears to be challenging as the shear
%velocity threshold tends to vary both spatially and temporally
%\citep{Barchyn2014b}. For example, soil moisture in the intertidal
%beach area fluctuates with the tidal phase and causes a local
%modulation of the shear velocity threshold. Moreover, a recurrence
%relation between sediment availability, and thus the shear velocity
%threshold, and sediment transport exists that complicates the a-priori
%determination of an appropriate threshold value. Consequently, aeolian
%sediment transport models tend to perform poorly in
%availability-limited systems.
%
%% what is done
%\citet{Sherman1998} and \citet{Sherman2012} summarized the performance
%of eight aeolian sediment transport models compared to field
%measurements on a sandy beach. Although it is unknown whether this
%coastal system was availability-limited, all models systematically
%overpredicted the measured aeolian sediment transport rates. This
%finding is in correspondence with an abundance of coastal field
%studies in which aeolian sediment transport rates are overestimated by
%numerical models \citep[e.g.][]{Jackson1999, Lynch2008,
%  DavidsonArnott2009, Aagaard2014}.
%
%In an attempt to explain the poor performance of aeolian sediment
%transport models in coastal environments, many authors emphasized the
%importance of sediment availability and bed surface
%properties. Typical bed surface properties that are found along the
%coast and known to affect sediment availability are high moisture
%contents \citep[e.g.][]{Wiggs2004, DavidsonArnott2008, Darke2008,
%  McKennaNeuman2008, Udo2008, Bauer2009, Edwards2009, Namikas2010,
%  Scheidt2010}, salt crusts \citep[e.g.][]{Nickling1981}, vegetation
%\citep[e.g.][]{Arens1996, Lancaster1998, Okin2008, Li2013,
%  Dupont2014}, shell pavements \citep[e.g.][]{VanDerWal1998,
%  McKennaNeuman2012} and sorted and armored beach surfaces
%\citep[e.g.][]{Gillette1989, Gillies2006, Tan2013, Cheng2015}. The
%influence of these bed surface properties on aeolian sediment
%availability and transport has been investigated and typically
%resulted in relations between bed surface properties and the shear
%velocity threshold \citep[e.g.][]{Howard1977, Dyer1986, Belly1964,
%  Johnson1965, Hotta1984, Nickling1981, Arens1996, King2005}.
%
%% what is needed
%Modeling rather than parameterization of spatiotemporal variations in
%aeolian sediment availability can improve coastal aeolian sediment
%transport estimates. As tides only affect the intertidal beach area,
%lag deposits and salt crusts typically emerge from the dry beach area,
%and vegetation is often restricted to the dune area, sediment
%availability varies spatially. In addition, temporal variations in
%sediment availability are induced by tidal spring/neap cycles, rain
%showers, storm surges, seasonal variations in vegetation and
%progressive armoring of the beach. Due to self-grading of the
%sediment, progressive beach armoring creates a recurrence relation
%between sediment availability and transport that challenges the
%a-priori determination of the spatiotemporal variations in sediment
%availability. Process-based modeling of the instantaneous shear
%velocity threshold field can address these challenges and improve
%coastal aeolian sediment transport estimates.
%
%% what did we do
%This paper presents the first application of a two-dimensional (2DH)
%aeolian sediment availability and transport model \citep{Hoonhout2016}
%to hindcast the development of the sub-aerial topography of an
%availability-limited coastal system. The model is unique in that it
%describes both spatial and temporal variations in aeolian sediment
%availability induced by the combined influence of sediment sorting,
%beach armoring and soil moisture content. The influence of
%spatiotemporal variations in aeolian sediment availability and the
%model performance are illustrated by a comparison between model
%results and a large scale sediment budgets analysis that identifies
%and quantifies the main sources and sinks for aeolian sediment in the
%coastal system \citep{Hoonhout2017a}.

%The \textsc{AeoLiS} model presented in Chapter \ref{ch:model} is
%applied to the Sand Motor field site as to hindcast the large scale
%aeolian sediment budgets discussed in Chapter \ref{ch:largescale}.
%The hindcast focuses on the effect of limitations in aeolian sediment
%availability that cause the sediment transport rates to be lower than
%the sediment transport capacity.

\section{Field Site}
\label{sec:fieldsite3}

The Sand Motor (or Sand Engine) is an artificial 21 $\mathrm{Mm^3}$
sandy peninsula protruding into the North Sea off the Delfland coast
in The Netherlands \citep[Figure \ref{fig:fieldsite3},][]{Stive2013}.
The Sand Motor was constructed in 2011 and its bulged shoreline
initially extended about 1 km seaward and stretched over approximately
2 km along the original coastline. The original coast was
characterized by an alongshore uniform profile with a vegetated dune
with an average height of 13 m and a linear beach with a 1:40
slope. The dune foot is located at a height of approximately 5 m+MSL.

\begin{figure}
  \centering
  \includegraphics[width=\columnwidth]{../Figures/location_and_evolution}
  \caption{Location, orientation, appearance and evolution of the Sand
    Motor between construction in 2011 and 2015. The box indicates the
    measurement domain used in the remainder of this paper. A 100 x
    100 m grid aligned with the measurement domain is plotted in gray
    as reference.}
  \label{fig:fieldsite3}
\end{figure}

Due to natural sediment dynamics the Sand Motor distributes about 1
$\mathrm{Mm^3}$ of sand per year to the adjacent coasts (Figure
\ref{fig:fieldsite3}). The majority of this sand volume is transported
by tides and waves. However, the Sand Motor is constructed up to 5
m+MSL and locally up to 7 m+MSL, which is in either case well above
the maximum surge level of 3 m+MSL (Figure
\ref{fig:boundaryconditions}c). Therefore, the majority of the Sand
Motor area is uniquely shaped by wind.

The Sand Motor comprises both a dune lake and a lagoon that act as
large traps for aeolian sediment (Figure \ref{fig:fieldsite3}). The
lagoon is affected by tidal forcing, although the tidal amplitude
quickly diminished over time as the entry channel elongated. The tidal
range of about 2 m that is present at the Sand Motor periphery (Figure
\ref{fig:boundaryconditions}c), is nowadays damped to less than 20 cm
inside the lagoon \citep{deVries2015}. Consequently, the tidal
currents at the closed end of the lagoon, where most aeolian sediment
is trapped, are negligible.

The dominant wind direction at the Sand Motor is south to southwest
(Figure \ref{fig:boundaryconditions}a). However, during storm
conditions the wind direction tends to be southwest to
northwest. During extreme storm conditions the wind direction tends to
be northwest. Northwesterly storms are typically accompanied by
significant surges as the fetch is virtually unbounded to the
northwest, while surges from the southwest are limited due to the
presence of the narrowing of the North Sea at the Strait of Dover
(Figure \ref{fig:fieldsite3}, inset).

\begin{figure}
  \centering
  \includegraphics[width=\columnwidth]{../Figures/boundaryconditions}
  \caption{Wind and hydrodynamic time series from 2011 to 2015. Hourly
    averaged wind speeds and directions are obtained from the KNMI
    meteorological station in Hoek van Holland (upper
    panels). Offshore still water levels, wave heights and wave
    periods are obtained from the Europlatform (lower panels). Runup
    levels are estimated following \citet{Stockdon2006}.}
  \label{fig:boundaryconditions}
\end{figure}

\section{Model description}

A two-dimensional (2DH) advection model for spatiotemporal varying
aeolian sediment transport and availability is used
\citep{Hoonhout2016}. The model simulates sediment availability
through the processes of sediment sorting, beach armoring and flooding
and drying. For this purpose the bed is discretized in horizontal grid
cells and in vertical bed layers (2DV). Moreover, the grain size
distribution is discretized into fractions. This allows the grain size
distribition to vary both horizontally and vertically.

The model describes the instantaneous sediment mass per unit area in
transport $c$ [$\mathrm{kg/m^2}$] by an advection equation, which
reads in one-dimensional notation:

\begin{equation}
  \label{eq:advection}
  \frac{\partial c_k}{\partial t} + u_z \frac{\partial c_k}{\partial x} = E_k - D_k
\end{equation}

\noindent where $t$ [s] denotes time, $x$ [m] denotes the cross-shore
distance from a zero-transport boundary, and $k$ [-] denotes the
sediment fraction index. $u_z$ [m/s] is the wind velocity at height
$z$ [m]. $E_k$ and $D_k$ [$\mathrm{kg/m^2/s}$] represent the erosion
and deposition terms and hence combined represent the net entrainment
of sediment.

The net entrainment is determined based on a balance between the
equilibrium or saturated sediment concentration $c_{\mathrm{sat},k}$
[$\mathrm{kg/m^2}$] and the instantaneous sediment transport
concentration $c_k$ and is maximized by the available sediment in the
bed $m_{\mathrm{a},k}$ [$\mathrm{kg/m^2}$] according to:

\begin{equation}
  \label{eq:erodep}
  E_k - D_k = \min \left ( \frac{\partial m_{\mathrm{a},k}}{\partial t} \quad ; \quad \frac{\hat{w}_k \cdot c_{\mathrm{sat},k} - c_k}{T} \right )
\end{equation}

\noindent where $T$ [s] represents an adaptation time scale that is
assumed to be equal for both erosion and deposition. $\hat{w}_k$ is a
weighting factor that sums to unity over the grain size fractions. The
saturated sediment concentration $c_{\mathrm{sat},k}$ is computed
using an empirical sediment transport formulation
\citep[e.g.][]{Bagnold1937a}.

The empirical sediment transport fomulation is provided with a term
for the shear velocity threshold $u_{\mathrm{* th},k}$ [m/s] that
defines the minimum wind shear required to initiate and sustain
saltation transport. The shear velocity threshold is determined based
on bed surface properties, like soil moisture content and the presence
of roughness elements.

Saturation of the soil is assumed to be instantaneous with rising
tide. The drying of the beach surface through infiltration is assumed
to follow an exponential decay. In order to capture this behavior the
volumetric water content is implemented according to:

\begin{equation}
  \label{eq:drying}
  p_{\mathrm{V}} = \left\{
    \begin{array}{ll}
      p & \mathrm{if} ~ \eta > z_{\mathrm{b}} \\
      p \cdot \int e^{\frac{\log \left( 0.5 \right)}{T_{\mathrm{dry}}} \cdot \mathrm{d} t} & \mathrm{if} ~ \eta \leq z_{\mathrm{b}} \\
    \end{array}
  \right.
\end{equation}

\noindent where $p$ [-] is the porosity, $\eta$ [m+MSL] is the
instantaneous water level, $z_{\mathrm{b}}$ [m+MSL] is the local bed
elevation, $p_{\mathrm{V}}$ [-] is the volumetric water
content. $T_{\mathrm{dry}}$ [s] is the beach drying time scale,
defined as the time in which the beach moisture content halves.

The sheltering effect of roughness elements protruding from the bed
and affecting the local wind shear and shielding local sediment is
implemented following \citet{Raupach1993}:

\begin{equation}
  \label{eq:raupach}
  f_{u_{\mathrm{* th, R}}} = \sqrt{ \left( 1 - m \sigma \lambda \right) \left( 1 + m \beta \lambda \right) }
\end{equation}

\noindent where $f_{u_{\mathrm{* th},R}}$ [-] is a factor with which
the local instantaneous shear velocity threshold per sediment fraction
is multiplied. $\lambda$ [-] is the roughness density. $m$, $\beta$
and $\sigma$ [-] are empirical factors defined in \citet{Raupach1993}
representing the difference between mean and maximum shear stress, the
ratio between the drag coefficient of the roughness elements alone and
the drag coefficient of the unarmored sandy bed, and the ratio between
the basal and frontal area of the roughness elements respectively.

\section{Model approach}

% link to sediment transport capacity
% link to specific results from chapter 2 for hindcast

The two-dimensional (2DH) model of the Sand Motor is constructed and
calibrated based on four years of field measurements on wind, tides,
waves and topography. The calibrated model is used to investigate the
influence of spatiotemporal variations in aeolian sediment
availability on sediment accumulation in the Sand Motor domain.

To test that the Sand Motor mega nourishment is indeed an
availability-limited coastal system, the measured long-term sediment
accumulation volumes \citep{Hoonhout2017a} are first compared to a
reference model that assumes no limitations in sediment availability
exist.

\subsection{Reference model}

\begin{figure}
  \centering
  \includegraphics[width=\columnwidth]{../Figures/transport_models}
  \caption{Comparison of the cumulative wind transport capacity
    according to a selection of equilibrium sediment transport
    formulations and measured total sedimentation in the Sand Motor
    domain. The equilibrium sediment transport is based on an hourly
    averaged wind speed and direction time series from September 1,
    2011 until September 1, 2015. Offshore wind directions are
    discarded. For the upper boundary of each estimate all wind
    directions are weighted equally. For the lower boundary of each
    estimate the wind directions are weighted according to the
    magnitude of the onshore component.}
  \label{fig:models}
\end{figure}

A selection of equilibrium sediment transport formulations is used as
reference model. An equilibrium sediment transport formulation
describes the wind transport capacity in given conditions. In
conjunction with a shear velocity threshold based on only a constant
uniform median grain size, an estimate of the potential aeolian
sediment accumulation in absence of availability-limitations can be
obtained. The potential aeolian sediment accumulation or cumulative
wind transport capacity $Q$ [$\mathrm{m^3}$] in the Sand Motor domain
is estimated based on hourly averaged time series of the wind speed
$u_z$ [m/s] and direction $\theta_u$ [$^{\circ}$] obtained from the
KNMI meteorological station in Hoek van Holland following:

\begin{equation}
  \label{eq:transport_capacity}
  Q = \sum q \cdot \frac{\Delta t \cdot \Delta y}{(1 - p) \cdot \rho_{\mathrm{p}}} \cdot f_{\theta_u}
\end{equation}

\noindent where the temporal resolution $\Delta t$ = 1 h, the
alongshore span of the domain $\Delta y$ = 4 km, the porosity $p$ =
0.4, the particle density $\rho_{\mathrm{p}}$ = 2650
$\mathrm{kg/m^3}$, the sediment transport rate $q$ is given by the
equilibrium sediment transport formulation (Table \ref{tab:models})
and $f_{\theta_u}$ is a factor to account for the wind direction. The
wind direction can be accounted for by only including the onshore wind
component with respect to the original coastline orientation. However,
given the typical Sand Motor geometry (Figure \ref{fig:fieldsite3}),
sediment is likely to be trapped in the dune lake and lagoon even with
alongshore wind. Therefore it can be assumed that the onshore wind
component will provide a lower limit of the cumulative wind transport
capacity. Similarly, an upper limit can be obtained by assuming that
all onshore wind directions contribute equally to the cumulative wind
transport capacity. For the upper limit the factor $f_{\theta_u}$ is
defined as:

\begin{equation}
  f_{\theta_u} = \left\{
      \begin{array}{rcl}
        1 & \mathrm{if} & \cos \left( 312\,^{\circ} - \theta_u \right) \geq 0 \\
        0 & \mathrm{if} & \cos \left( 312\,^{\circ} - \theta_u \right) < 0 \\
      \end{array}
    \right.
\end{equation}

\noindent while for the lower limit the factor $f_{\theta_u}$ is defined
as:

\begin{equation}
  f_{\theta_u} = \max \left( 0 \quad ; \quad \cos \left( 312\,^{\circ} - \theta_u \right) \right)
\end{equation}

\noindent where $312\,^{\circ}$ accounts for orientation of the original
coastline.  Figure \ref{fig:models} presents an overview of the
cumulative wind transport capacity in the Sand Motor domain over the
period between September 1, 2011 and September 1, 2015 according to a
selection of equilibrium sediment transport formulations and in
comparison with the measured accumulation volumes. The estimates of
the wind transport capacity show a large variation between
formulations that are mainly due to the incorporation of the shear
velocity threshold. However, all formulations overestimate the
measured sediment accumulation in the Sand Motor domain with at least
a factor 3 -- 4. The large variation and consistent overestimation is
in accordance with the review of aeolian sediment transport models
presented by \citet{Sherman2012}. The consistent overestimation of the
measured sedimentation volumes in the Sand Motor domain suggest that
the Sand Motor is indeed an availability-limited coastal system.

\subsection{Schematization}

\begin{figure}
  \centering
  \includegraphics[width=\columnwidth]{../Figures/gridmask}
  \caption{Model grid and topography based on the topographic survey
    of August 3, 2011 (upper panel) and hydrodynamic mask used to
    limit tidal and wave motions in the dune lake and lagoon (middle
    and lower panels). Water levels and wave heights are uniformly
    imposed to the model and multiplied by the multiplication mask and
    subsequently increased with the addition mask.}
  \label{fig:gridmask}
\end{figure}

% topography and grid
A two-dimensional (2DH) aeolian sediment availability and transport
model for the Sand Motor mega nourishment is constructed for the four
years between September 1, 2011 and September 1, 2015, which is
shortly after the nourishment was placed. The model's topography and
grid are based on the measured topographies of August 3, 2011 and
later. The topographies are rotated $48\,^{\circ}$ and interpolated to
a 50 x 50 m grid spanning 1.5 km cross-shore and 4 km alongshore with
respect to the original coastline, not including the dunes (Figure
\ref{fig:gridmask}, upper panel).

% boundary conditions
Four years of hourly wind speed and direction data measured at 10 m
above the bed is obtained from the KNMI meteorological station at Hoek
van Holland (Figure \ref{fig:boundaryconditions}a,b). Hourly offshore
water levels and wave heights are obtained from the Europlatform for
the same period (Figure \ref{fig:boundaryconditions}c,d).

% grain size and shells, layers and layer thickness
An average lognormal grain size distribution with a median diameter
$d_{50} = 335 ~ \mu \mathrm{m}$ is used as measured at the Sand Motor
field site. The sand fractions cover a range from 0.1 to 2 mm. The
amount of shells and other roughness elements in the originally
nourished sand is estimated to be 5\%. The estimate is based on three
sediment samples obtained from the field site 0.5 m below the bed
surface. Additional fractions ranging from 2 to 32 mm are added
according to a lognormal distribution to account for the presence of
roughness elements in the bed. The grain size distribution is used to
populate the initial bed that consists of 10 bed composition layers
with a thickness of 1 cm each.

% timestep and scheme
The hindcast aims at the large scale and long term sedimentation
volumes as presented by \citet{Hoonhout2017a}. Therefore an efficient,
but diffusive, implicit Euler Backward scheme with a timestep of 1 h
is used that does not resolve high frequency variations in wind or
sediment transport. Consequently, the model produces smooth solutions
that describe hourly steady states based on the instantaneous average
wind speed and sediment availability.

% equilibrium sediment transport formulation
\begin{table}
  \centering
  \caption{Equilibrium sediment transport formulations, coefficient
    values* and the ratio between measurements and model results.}
  \label{tab:models}
  \begin{tabular}{llll}
    Reference & Equation & $C$ & Ratio \\
    \hline
    \citet{Bagnold1937a} & $q = C \frac{\rho_{\mathrm{a}}}{g} \sqrt{\frac{d_{\mathrm{n}}}{D_{\mathrm{n}}}} \left(u_{\mathrm{*}} - u_{\mathrm{* th}} \right)^3$ & 1.8 & 3 -- 4 \\
    \citet{Horikawa1983} & \multirow{2}{*}{$q = C \frac{\rho_{\mathrm{a}}}{g} \left(u_{\mathrm{*}} + u_{\mathrm{* th}} \right)^2 \left(u_{\mathrm{*}} - u_{\mathrm{* th}} \right)$} & 1.0 & 5 -- 8 \\
    \citet{Kawamura1951} &  & 2.78 & 14 -- 22 \\
    \citet{Lettau1978} & $q = C \frac{\rho_{\mathrm{a}}}{g} \sqrt{\frac{d_{\mathrm{n}}}{D_{\mathrm{n}}}} \left(u_{\mathrm{*}} - u_{\mathrm{* th}} \right) u_{\mathrm{*}}^2$ & 6.7 & 46 -- 75 \\
  \end{tabular}

  \footnotesize{
    \begin{enumerate}[{*}]
    \item Other values are the shear velocity
      $u_{\mathrm{*}} = \alpha \cdot u_z$ m/s, the shear velocity
      threshold $u_{\mathrm{* th}} = \alpha \cdot 3.87$ m/s, the
      conversion factor from free-flow wind velocity to shear velocity
      $\alpha = 0.058$, the air density $\rho_{\mathrm{a}}$ = 1.25
      $\mathrm{kg/m^3}$, the particle density $\rho_{\mathrm{p}}$ =
      2650.0 $\mathrm{kg/m^3}$, the gravitational constant $g$ = 9.81
      $\mathrm{m/s^2}$, the nominal grain size $d_{\mathrm{n}}$ = 335
      $\mu \mathrm{m}$, a reference grain size $D_{\mathrm{n}}$ = 250
      $\mu \mathrm{m}$ and the height above the bed of the wind
      measurement $z$ = 10 m.
    \end{enumerate}
  }
\end{table}

\citet{Bagnold1937a} is selected as equilibrium sediment transport
formulation as it is derived separately for different grain sizes and
therefore suitable for multi-fraction aeolian sediment
transport. Alternative formulations (Table \ref{tab:models}) are
derived for wider grain size distributions that do not necessarily
result in a monotonic relation between the grain size and the sediment
transport rate \citep[e.g.][]{Kawamura1951, Horikawa1983}. Such
non-monotonic relation is unrealistic in a multi-fraction context as
it would result in a preference to transport both fine sediment and
large elements that are considered non-erodible. Moreover, the
formulation of \citet{Bagnold1937a} overestimates the measured aeolian
sediment transport rates in the Sand Motor domain less compared to
alternative formulations (Table \ref{tab:models}, rightmost column).

% masks
Water levels and wave heights are initially uniformly imposed to the
model. Consequently, the tidal range, mean water level and wave
heights that are present at the Sand Motor periphery are also present
in the dune lake and lagoon. In reality, the tidal range and wave
heights in the dune lake and lagoon are much lower, while the mean
water level in the dune lake and lagoon is elevated compared to mean
sea level \citep{deVries2015}. To account for these spatial
differences in hydrodynamics a hydrodynamic mask is applied (Figure
\ref{fig:gridmask}, middle and lower panel).

% BMI
Subtidal changes in topography are not simulated by the model. The
subtidal changes can be important to aeolian sediment transport as the
location and size of aeolian sediment erosion and deposition areas
might change. To account for these changes, measured topographies are
imposed to the model through a Basic Model Interface
\citep[BMI,][]{Peckham2013}.

All measured topographies in the period between September 1, 2011 and
September 1, 2015 are linearly interpolated in time as to obtain daily
updates of the Sand Motor's topography. The hydrodynamic mask is
updated along with the topography. The presented aeolian sediment
transport rates are based on the time-integrated entrainment and
deposition rates that are computed by the model rather than
differences in topography.

\subsection{Calibration}

The model is calibrated on the shape of roughness elements that emerge
from the bed and shelter the sand surface from wind erosion, the
drying rate of the soil and the time needed for the sediment transport
to adapt to changing wind conditions. These processes are represented
in the model by parameters for which data or literature can only
provide approximate values:

\begin{enumerate}
\item $\sigma$, as used in the formulation of \citet[][Equation
  \ref{eq:raupach}]{Raupach1993}, is the ratio between the basal and
  frontal area of the roughness elements that constitute the beach
  armor layer.
\item $T_{\mathrm{dry}}$ is the time scale at which the beach dries
  out after flooding (Equation \ref{eq:drying}). It represents the
  time in which the soil moisture content halves in case the beach is
  not inundated and no evaporation occurs.
\item $T$ is the adaptation time scale in the right-hand side of the
  advection equation (Equation \ref{eq:erodep}). It represents the
  time scale to which the sediment transport adapts to variations in
  the wind conditions and sediment availability.
\end{enumerate}

% Wind tunnel experiments presented by \citet{McKennaNeuman2012}
% provide leads on appropriate values for $\sigma$. The wind tunnel
% experiments investigated the influence of a beach armor layer,
% consisting of shells and shell fragments, on aeolian sediment
% transport. The results were used to derive optimal values for the
% parameters $\sigma$ and $\beta$ (Equation \ref{eq:raupach}).

The implementation of roughness elements is characterized by three
calibration parameters: $m$, $\beta$ and $\sigma$ (Equation
\ref{eq:raupach}). $m$ is a factor to account for the difference
between the mean and maximum shear stress and is usually chosen as 0.5
for field applications \citep{Raupach1993,
  McKennaNeuman2012}. Numerically it is irrelevant if $\beta$ or
$\sigma$ is calibrated as they only appear as a ratio
$\frac{\beta}{\sigma}$ in the model implementation. As $\beta$ is the
ratio between the drag coefficient of the roughness elements alone and
the drag coefficient of the unarmored sandy bed, the value can be
assumed to be reasonably generic. In contrast, $\sigma$ depends on the
shape and protrusion of the roughness elements and therefore depends
on the field site and varies in time. For example, a spherical object
placed on top of the bed would be represented by $\sigma = 1$, while a
spherical object protruding halfway through the bed (hemisphere) would
be represented by $\sigma = 2$. Consequently, calibration of $\sigma$
seems to be preferable as it is less certain. Wind tunnel experiments
presented by \citet{McKennaNeuman2012} investigated the influence of
lag deposits, consisting of shells and shell fragments, on aeolian
sediment transport. Values for the calibration coefficients $m$ and
$\beta$ were found to be 0.5 and 130 respectively and are adopted for
the Sand Motor hindcast. An optimal average value for $\sigma$ is
obtained by systematic variation between 2 and 20.

The drying rate of the beach ($T_{\mathrm{dry}}$) depends on many
factors, like grain size, soil moisture content, groundwater level,
wind speed and solar radiation. The use of a single time scale as
aggregate for these processes is an oversimplification of
reality. Therefore a wide range of parameter values is covered in the
calibration. $T_{\mathrm{dry}}$ is varied between 0.1 and 10 hours
where the former results in virtually instant drying and the latter
results in an intertidal beach that is permanently too moist for
aeolian sediment transport to be initiated.

The adaptation time scale ($T$), that represents the swiftness of
aeolian sediment transport to adapt to changing wind conditions, is in
the order of seconds \citep{DavidsonArnott2008, deVries2014a}. As the
model time step is orders of magnitude larger, the model effectively
solves steady states and the value for $T$ will not affect temporal
variations in sediment transport. However, the adaptation time scale
also affects the development of the saltation cascade in
space. Sediment transport increases in downwind direction from a
zero-flux boundary, like the water line in case of onshore wind, with
a rate that is governed by the value of $T$. Consequently, $T$
influences the width of the source area in case of abundant sediment
availability. $T$ is varied between 1 and 10 seconds.

\begin{figure}
  \centering
  \includegraphics[height=12cm, angle=90]{../Figures/decomposition}
  \caption{Zonation of the Sand Motor domain into zones with net
    aeolian erosion and no marine influence, net aeolian deposition
    and no marine influence, mixed aeolian/marine influence and marine
    influence. Zonation is based on the 0, 3 and 5 m+MSL contour lines
    that roughly correspond with the mean water level, maximum runup
    level or berm edge and the dune foot respectively. Left panels:
    2011. Right panels: 2015. Source: \citet{Hoonhout2017a}.}
  \label{fig:decomposition2}
\end{figure}

The calibration is performed based on the bi-monthly erosion and
deposition volumes as measured in the Sand Motor domain
\citep{Hoonhout2017a}. The erosion and deposition volumes are
determined within seven predefined zones (Figure
\ref{fig:decomposition2}) that aim to separate areas with marine
influences (marine zone, mixed zone and lagoon) from areas without
marine influences (aeolian zone, dune lake and dunes), and separate
areas with net aeolian erosion (mixed and aeolian zone) from areas
with net aeolian deposition (lagoon, dune lake and dunes). The zonation
is based on the 0, 3 and 5 m+MSL contour lines that roughly correspond
with the mean water level, maximum runup level or berm edge and the
dune foot respectively. The average $\mathrm{R^2}$ value of the time
series for erosion and deposition is used as benchmark. The
$\mathrm{R^2}$ value represents the fraction of explained variance and
is defined as:

\begin{equation}
  \label{eq:r2}
  R^2 = 1 - \frac{\sum_n \left[ V^n_{\mathrm{measured}} - V^n_{\mathrm{model}} \right]^2}{\sum_n \left[ V^n_{\mathrm{measured}} - \overline{V^n_{\mathrm{measured}}} \right]^2}
\end{equation}

\noindent where $V^n$ is the measured or modeled sediment volume in
time period $n$. The overbar denotes time-averaging. In addition the
root-mean-square error (RMSE) is presented as absolute measure for the
model accuracy, which is defined as:

\begin{equation}
  \label{eq:rmse}
  RMSE = \sqrt{\sum_n \left[ V^n_{\mathrm{measured}} - V^n_{\mathrm{model}} \right]^2}
\end{equation}

\noindent The calibration itself is performed in three steps:

\begin{enumerate}
\item A coarse calibration on $\sigma$ and $T_{\mathrm{dry}}$.
\item A calibration on $T$ using the provisional optimal settings for
  $\sigma$ and $T_{\mathrm{dry}}$.
\item A fine calibration on $\sigma$ and $T_{\mathrm{dry}}$ using the
  optimal setting for $T$.
\end{enumerate}

\section{Results}

The optimal model settings were chosen from 150 realizations (Figure
\ref{fig:calibration}). The optimal realization has an $\mathrm{R^2}$
value of 0.93 and a RMSE of $3 \cdot 10^4 ~ \mathrm{m^3}$.
The corresponding optimal parameter settings are found to be
$\sigma = 9.2$, $T_{\mathrm{dry}} = 2$ h and $T$ = 1 s. These
settings were ultimately selected from a cluster of realizations with
comparable $\mathrm{R^2}$ values based on the relative sediment supply
from the mixed zones (Figure \ref{fig:decomposition2}, third row) at
the end of the simulation. An overview of all model settings for the
calibrated model is given in Appendix \ref{apx:modelsettings}.

\begin{figure}
  \centering
  \includegraphics[width=\columnwidth]{../Figures/calibration}
  \caption{Systematic variation of calibration parameters $\sigma$ and
    $T_{\mathrm{dry}}$ with $T$ = 1 s. The circles indicate the
    realizations made. The colored background depicts a linear
    interpolation of the $\mathrm{R^2}$ values with respect to the
    measurement data presented in \citet{Hoonhout2017a} and Figure
    \ref{fig:netvolumechange_model}. The solid isolines depict
    $\mathrm{R}^2$ values from 0.90 to 0.93, while the dashed isolines
    depict $\mathrm{R}^2$ values from 0.0 to 0.9. The red lines depict
    the relative supply from the mixed zones ranging from 52\% to
    57\%. The yellow star indicates the optimal value model settings.}
  \label{fig:calibration}
\end{figure}

\begin{figure}
  \centering
  \includegraphics[height=15cm, angle=90]{../Figures/model_sedero}
  \caption{Simulated and measured yearly sedimentation and erosion
    above 0 m+MSL. Model results only include aeolian sediment
    transport as hydrodynamic sediment transport is not
    computed. Comparisons are made between the September surveys of
    each year.}
  \label{fig:sedero_model}
\end{figure}

\begin{figure}
  \centering
\includegraphics[width=\columnwidth]{../Figures/model_volumes_ts}
\caption{Simulated net volume change of erosion and deposition volumes
  compared to measured net volume change as presented in
  \citet{Hoonhout2017a}.}
  \label{fig:netvolumechange_model}
\end{figure}

\begin{figure}
  \centering
  \includegraphics[width=\columnwidth]{../Figures/model_volumes}
  \caption{Total erosion and deposition volumes at the end of the
    simulation and measured total erosion and deposition volumes as
    presented in \citet{Hoonhout2017a}.}
  \label{fig:volumes_bars_model}
\end{figure}

\begin{figure}
  \centering
  \includegraphics[width=\columnwidth]{../Figures/model_heights}
  \caption{Simulated average beach height in the aeolian zone compared
    to measured average beach height as presented in
    \citet{Hoonhout2017a}.}
  \label{fig:heights_model}
\end{figure}

Figure \ref{fig:sedero_model} shows that erosion from the aeolian zone
(Figure \ref{fig:decomposition2}, first row) is most pronounced in the
first year and least in the second year in both the measurements and
the model results. Also the deposition of aeolian sediment in the dune
lake and lagoon (Figure \ref{fig:decomposition2}, second row) is
observed in both the measurements and model results, although the
model underestimates these deposited volumes. The deposition in the
dune lake and lagoon is also more localized in the measurements than
in the model results. The spatial variability in the erosion of the
aeolian zone is larger in the measurements than in the model
results. The large variability measured in the mixed zone is not
present in the model results as hydrodynamic sediment transport is not
simulated.

The development of the total erosion and deposition volumes in the
Sand Motor domain in the four year period is represented well by the
model (Figure \ref{fig:netvolumechange_model}). The dune accumulation
volume is overestimated at the expense of the sediment volumes
deposited in the dune lake and lagoon (Figure
\ref{fig:volumes_bars_model}). As the dune area is not included in the
model domain, the sediment flux over the onshore boundary is assumed
to settle in the dunes entirely. The total sediment accumulation at
the end of the simulation is underestimated by 12\% as the offshore
sediment deposits are not included in the large scale sediment budget
analysis that are used for comparison. The underestimation is unique
for the last nine months of the simulation as the model overestimates
the total sediment accumulation with 5\% on average (Figure
\ref{fig:netvolumechange_model}). The relative importance of the mixed
zone as supplier of aeolian sediment is well captured.

The change in beach height within the most recent 3 m+MSL contour,
that marks the aeolian zone, is represented by the model as the
$\mathrm{R^2}$ value is 0.71 and the RMSE is about 4 cm or 12\% of the
average bed level change (Figure \ref{fig:heights_model}). As the
change in beach height is computed within the most recent 3 m+MSL
contour, the discrepancy is illustrative for the differences in
spatial variability in erosion between measurements and model
results. The lowering of the beach in the aeolian zone in the first
half year of the simulation is particularly underestimated, while the
accelerated erosion in this period is well captured in the total
sediment transport. This indicates that sediment is eroded from
outside the most recent 3 m+MSL contour.

The coverage of non-erodible elements $\lambda \sigma$ [-] (Equation
\ref{eq:raupach}) in the aeolian zone varies between 60\% and 80\% at
the end of the simulation. The coverage is high compared to the 10\%
-- 20\% shell coverage estimated to be present at the Sand Motor above
3 m+MSL based on gridded photographs.
% (Figure \ref{fig:armoring}).

\begin{figure}
  \centering
  \includegraphics[width=\columnwidth]{../Figures/space_vs_time}
  \caption{The influence of time-varying and space-varying shear
    velocity thresholds on the total sedimentation volume. The two
    leftmost bars depict the measured and modeled sedimentation volume
    as obtained from the calibrated model (Figure
    \ref{fig:volumes_bars_model}). The middle two bars depict results
    from two separate model simulations in which a space-averaged
    threshold time series or a time-averaged threshold field is
    imposed respectively. The former neglects compartmentalization,
    the latter neglects periodic flooding. The threshold averages are
    based on the result from the calibrated model. The two rightmost
    columns depict a result from a separate model simulation with a
    constant uniform threshold based on only a constant uniform median
    grain size and the estimated equilibrium sediment transport
    following \citet{Bagnold1937a} respectively (Table
    \ref{tab:models}).}
  \label{fig:space_vs_time}
\end{figure}

Both the spatial and temporal variations in aeolian sediment
availability are crucial for an accurate description of the
compartmentalization of the Sand Motor. Compartmentalization governs
both the total sedimentation and erosion volumes and the location of
the aeolian sediment source and deposition areas. Figure
\ref{fig:space_vs_time} compares the total sedimentation volume
according to measurements, the calibrated model and additional
simulations, that are variations of the calibrated model in which
spatial and/or temporal variations in the shear velocity threshold are
averaged out. During these additional simulations the shear velocity
threshold is not computed by the model, but space- and/or
time-averaged thresholds based on the model results of the calibrated
model are imposed. Negligence of the spatial variations results in an
absence of compartmentalization and a 79\% underestimation of the
total sedimentation volume and a relative contribution of 8\% of the
mixed zones. The negligence of the temporal variations results in
conservation of the compartmentalization, but an increased
contribution of 86\% of the mixed zones. The total sedimentation
volume is 46\% overestimated in this case. In addition, a simulation
without limitations in sediment availability overestimates the
measured total sedimentation volumes with 400\%, which is comparable
to the wind transport capacity following \citet[][Figure
  \ref{fig:models}]{Bagnold1937a}.

\section{Discussion}

The model results show that multi-annual aeolian sediment erosion and
deposition volumes, and the relative importance of the mixed zones as
source of aeolian sediment are reproduced with reasonable
accuracy. This suggests that indeed significant limitations in
sediment availability, due to soil moisture content and beach
armoring, govern aeolian sediment transport in the Sand Motor
domain. A comparison with a simulation without limitation in sediment
availability suggests that aeolian sediment availability in the Sand
Motor domain is limited to about 25\% -- 35\% of the wind transport
capacity.

The negligence of spatial variations causes the model to underestimate
the measured total sedimentation volume. The sediment supply from the
relatively small mixed zone is marginalized as the imposed
space-averaged shear velocity threshold is relatively high. In
contrast, the negligence of temporal variations causes the model to
overestimate the measured total sedimentation volume. The sediment
supply from the mixed zones is increased as the effect of its periodic
flooding is averaged out. At the same time, the sediment supply from
the aeolian zone is decreased as the influence of beach armoring
affects sediment availability from the start of the simulation rather
than after the development of the beach armor layer. Therefore, the
total sedimentation volume is not only overestimated, but also the
importance of the mixed zones as supplier of aeolian sediment. This
suggests that compartmentalization indeed govern the location of the
aeolian sediment source and deposition areas and the total sediment
accumulation in the Sand Motor domain.

\subsection{Seasonal and local variations in sedimentation and
  erosion}

% advection assumption / diffusion / gusts
The model can reproduce multi-annual trends in sedimentation volume,
which is the aim of the hindcast, but seasonal and local variations
are sometimes not represented by the model. An analysis of these
variations is interesting as they influence the accuracy of specific
model results.

Average wind speeds tend to be elevated in December and January
(Figure \ref{fig:boundaryconditions}), which leads to short periods of
accelerated sediment accumulation in the beginning of 2012, 2013 and
2015 that are captured well by the model. Early 2014 no accelerated
sediment accumulation is measured, while the model simulation shows an
increase in sediment accumulation originating from the mixed zones
similar to other years.

The discrepancy early 2014 might be explained by topographic changes
induced by hydrodynamic forces. On December 5th, 2013 an exceptional
storm hit the Dutch coast. During this storm a significant decrease in
aeolian deposits in the lagoon was observed, while deposits in the
dunes and dune lake increased only marginally. The assumption that the
closed end of the lagoon is mainly governed by aeolian sediment
transport might be violated in these exceptional conditions. At the
same time, the erosion of the aeolian zone that day equaled the total
erosion of the aeolian zone that year. Consequently, the total
subaerial sediment volume decreased that day with about $\mathrm{1
  \cdot 10^4}$ $\mathrm{m^3}$, possibly caused by hydrodynamic
forces. This suggests that the simplified hydrodynamics, despite the
use of a hydrodynamic mask, are a limiting factor in describing the
Sand Motor's subaerial morphodynamics during extreme storms.

In the first months of the simulation, the total sediment accumulation
is well represented, but erosion of the aeolian zone is
underestimated. As beach armoring is the most important availability
limitation in the aeolian zone, this suggests that the armoring rate
is overestimated by the model. The armoring rate is mainly influenced
by initial shell fraction of 5\%, which might be
overestimated. Alternatively, the initially uniform distribution of
shells in the bed is not an accurate representation of reality.

Measured erosion and deposition rates exceed modeled erosion and
deposition rates in the final nine months of the simulation. In this
period dune growth seems to accelerate, while neither the deposition
in the dune lake and lagoon did accelerate nor did the wind speed
increase. The apparent acceleration is therefore solely found in the
half yearly lidar measurements of the dune area \citep{Hoonhout2017a}
and is consequently based on a single data point. Despite the
uncertainty involved in the measured acceleration, also precipitation
rates, that were up to 70\% lower in this period compared to the same
period in other years, might explain the discrepancy at the end of the
simulation \citep{Jackson1998}. For the hindcast no precipitation time
series are imposed as the effect on the aeolian sediment transport
rate is not properly understood yet. Consequently, the calibration of
the model might have resulted in an overestimated importance of beach
armoring to compensate for the negligence of precipitation.

% morphological feedback
The distribution of the aeolian sediment deposits over the dune lake,
lagoon and dunes is not represented well as deposits in the dune lake
and lagoon are underestimated. Additional hydrodynamic and hydrologic
processes, like wind setup and groundwater seepage, might cause the
entrapment area in reality to be larger than modeled. But more
importantly, the dune lake and lagoon are positioned in the lee of the
Sand Motor crest with respect to the predominant southwesterly wind
direction. The height difference between the Sand Motor crest and the
water level in the lagoon and dune lake is several meters, which is
likely to influence the local wind field significantly. The probable
decrease in wind shear in the lee of the Sand Motor crest promotes
deposition of aeolian sediment and likely hampers supply to the
dunes. These local variations in wind shear are not included in the
simulations.

\subsection{Beach armoring, sediment availability and the shear
  velocity threshold}

% As the $\sigma$ value is an optimal average for the entire model
% domain and duration it can be questioned if an incidental increase
% in soil moisture due to precipitation can influence sediment supply
% significantly.

\begin{figure}
  \centering
\includegraphics[width=\columnwidth]{../Figures/sigma}
\caption{Relation between shear velocity threshold, shell coverage and
  $\sigma$ according to \citet[][Equation
  \ref{eq:raupach}]{Raupach1993}. The shaded areas indicate the
  relevant parameter ranges from \citet{McKennaNeuman2012} (blue) and
  the model results (green).}
  \label{fig:sigma}
\end{figure}

The influence of beach armoring is reflected in the model by both
$\sigma$ and the roughness density $\lambda$ (Equation
\ref{eq:raupach}). The optimal value for $\sigma$ was found to be 9.2,
which is high compared to the value of 4.2 found by
\citet{McKennaNeuman2012}. The difference suggests that the roughness
elements at the Sand Motor protrude less from the bed compared to what
was found in the wind tunnel experiments. Consequently, the importance
of beach armoring would be relatively low and compartmentalization of
limited influence at the Sand Motor. However, the low $\sigma$ value
is largely compensated by the roughness density $\lambda$ reflected in
a shell coverage $\sigma \lambda$ that is high compared to what was
found in the wind tunnel experiments (12\% -- 43\% on average) and
what is found at the Sand Motor field site (10\% -- 20\%). Figure
\ref{fig:sigma} shows that the combination of high shell coverage and
$\sigma$ value results in a very similar increase of the shear
velocity threshold compared to the wind tunnel experiments presented
by \citet{McKennaNeuman2012}.

The reason that the model calibration resulted in this particular
value for $\sigma$ is that the model does not differentiate between
the fluid and impact velocity threshold.  Therefore, the roughness
elements in the model affect the initiation of sediment transport
equal to the continuation of sediment transport. The potential
reduction in sediment availability increases with a decreasing value
for $\sigma$ (if $m$ = 0.5, Figure \ref{fig:sigma}) and is implemented
through an increase in shear velocity threshold. The shear velocity
threshold also affects aeolian sediment already in transport and
originating from upwind, unarmored beach areas, like the mixed
zones. Sediments from upwind areas are therefore partially deposited
in the aeolian zone as soon a beach armor layer develops. For low
values for $\sigma$ the local deposition of sediment from upwind areas
is already significant with low shell coverage. Low $\sigma$ values
therefore reduce the total sediment accumulation in the dunes
quickly. In order for the model to provide reasonable total sediment
transport rates, a higher value for $\sigma$ was found in the
calibration that ultimately induces a higher shell coverage. The value
for $\sigma$ therefore does not only represent a spatiotemporal
averaged emergence of roughness elements, but also a compromise
between its effect on the fluid and impact velocity threshold.

Note that the model conceptually allows to differentiate between the
impact and fluid threshold. The right-hand side of the advection
equation (Equation \ref{eq:erodep}) can be modified according to:

\begin{equation}
  \label{eq:erodep_split}
  E_k - D_k = \min \left( 
  \frac{\partial m_{\mathrm{a},k}}{\partial t} \quad ; \quad 
  \frac{\hat{w}_k}{T} \cdot \left[
    (1 - S_k) \cdot c^{\mathrm{fluid}}_{\mathrm{sat},k} +
    S_k \cdot c^{\mathrm{impact}}_{\mathrm{sat},k} - c_k
    \right]
  \right)
\end{equation}

\noindent where $c^{\mathrm{fluid}}_{\mathrm{sat},k}$
[$\mathrm{kg/m^2}$] and $c^{\mathrm{fluid}}_{\mathrm{sat},k}$
[$\mathrm{kg/m^2}$] are the sediment transport capacity associated
with the fluid and impact threshold respectively and $S_k$ [-] is the
degree of saturation.

Unfortunately, empirical data to quantify the differentiation is
lacking. This potential model improvement is therefore still
hypothetical and requires fundamental research on the impact and fluid
shear velocity threshold under varying conditions.



% average wind vs gusts

% wind field

% bed composition layers

% offshore deposits

% salt crusts

% distinction between thresholds for entrainment and continuation

\section{Conclusions}

The hindcast of the morphological development of the Sand Motor mega
nourishment shows that the aeolian sediment transport and availability
model \textsc{AeoLiS} captures the particular supratidal morphodynamic
behaviour associated with mega nourishments. The supratidal
morphodynamic behaviour at the Sand Motor is characterized by the
significant compartmentalization, modest aeolian sediment transport
rates and relatively low dune growth rates. The reduction of aeolian
sediment availability due to soil moisture and beach armoring in the
aeolian zone results in compartmentalization.  Compartmentalization
explains the major importance of the intertidal beach area as supplier
of aeolian sediment and can largely explain the relatively low
accumulation volumes in the Sand Motor domain when compared to
adjacent coasts. The model also reflects the minor importance of the
beach widths up to 1 kilometer that are present at the Sand Motor.

\bigskip

\noindent From the hindcast the following conclusions can be drawn:

\begin{itemize}
\item The \textsc{AeoLiS} model is able to reproduce multi-annual
  aeolian sediment transport rates in the Sand Motor domain in the
  four years after its construction with a RMSE of $3 \cdot 10^4 ~
  \mathrm{m^3}$ (7\% of the total sediment accumulation of $40 \cdot
  10^4 ~ \mathrm{m^3}$) and $\mathrm{R^2}$ of 0.93 when time series of
  measured and modeled total aeolian sediment transport volumes are
  compared.
\item The \textsc{AeoLiS} model is able to reproduce large scale
  spatial patterns in aeolian sediment transport in the Sand Motor
  domain in the four years after its construction, but underestimates
  the deposition in the dune lake and lagoon, likely due to wind setup
  and groundwater seepage that are not yet included in the model.
\item The \textsc{AeoLiS} model overestimates the total sedimentation
  volume with 5\% on average, but underestimates the total
  sedimentation volume with 12\% at the end of the simulation. The
  discrepancy at the end of the simulation might be caused by a
  particularly dry season as precipitation is not enabled in the
  simulations.
\item The \textsc{AeoLiS} model is able to capture the seasonal
  variations in sediment transport in all years, except for early 2014
  when significant morphological change is possibly related to
  hydrodynamic sediment transport that is not included in the
  simulations.
\item The \textsc{AeoLiS} model overestimates the shell coverage,
  which compensates the high value for $\sigma$. The high $\sigma$
  value is a compromise between the fluid and impact threshold that
  are currently assumed to be equal.
\end{itemize}

\noindent From the hindcast the following general conclusions can be
drawn regarding spatiotemporal variations in aeolian sediment
availability and transport and beach compartmentalization:

\begin{itemize}
\item The combination of spatial and temporal variations in aeolian
  sediment availability, due to the combined influence of soil
  moisture, sediment sorting and beach armoring, and the feedback
  between aeolian sediment availability and transport is essential for
  an accurate estimate of the total sedimentation volume at coastal
  sites where beach compartmentalization is present.
\item Compartmentalization of the beach due to beach armoring should
  be taken into account when designing (mega) nourishments as it
  can govern aeolian sediment availability and transport.
\item Compartmentalization of the beach can influence the lifetime and
  region of influence of a (mega) nourishments as it governs the
  wind-driven erodibility of the mega nourishment.
\end{itemize}

%%% Local Variables:
%%% mode: latex
%%% TeX-master: "paper_hindcast_ce"
%%% End:


\section*{Acknowledgements}
The work discussed in this paper is supported by the ERC-Advanced
Grant 291206 -- Nearshore Monitoring and Modeling (NEMO) and Deltares.

\appendix

\ifthenelse{\boolean{ispaper}}{
  \section{Model settings}
  \label{apx:modelsettings}

  The model schematizations presented in this paper used the settings
  listed below. Some model settings belong to experimental features of
  the model and are not discussed. These settings are listed for
  completeness only and marked with an asterisk (*). The model
  settings are chosen such that experimental features are disabled.
}{
  \chapter{Model settings}
  \label{apx:modelsettings}

  Unless stated otherwise, the model schematizations presented in
  chapter \ref{ch:hindcast} used the settings listed below. Some model
  settings belong to experimental features of the model and are not
  discussed in this thesis. These settings are listed for completeness
  only and marked with an asterisk (*). The model settings are chosen
  such that experimental features are disabled.
}

\begin{longtable}{p{3cm} p{8cm}}
  Parameter         & Value \\
  \hline
  \endfirsthead
  Parameter         & Value \hfill (continued)\\
  \hline
  \endhead
  A                 & 0.085 \\
  CFL               & 1.0 \\
  Cb                & 1.5 \\
  T                 & 1.0 \\
  Tdry              & 5400.0 \\
  Tsalt*            & 0.0 \\
  accfac            & 1.0 \\
  bedupdate         & False \\
  beta              & 130.0 \\
  bi                & 0.05 \\
  boundary\_lateral  & circular \\
  boundary\_offshore & noflux \\
  boundary\_onshore  & gradient \\
  callback          & None \\
  cpair             & 0.0010035 \\
  csalt*            & 0.035 \\
  dt                & 3600.0 \\
  eps               & 0.001 \\
  evaporation       & True \\
  facDOD            & 0.1 \\
  g                 & 9.81 \\
  gamma             & 0.5 \\
  grain\_dist       & 0.005709 0.234708 0.608887 0.099666 0.001029 0.000001 0.010486 0.028503 0.010486 0.000522 0.000004 \\
  grain\_size       & 0.000177 0.000250 0.000354 0.000500 0.000707 0.001000 0.002000 0.004000 0.008000 0.016000 0.032000 \\
  k                 & 0.01 \\
  layer\_thickness  & 0.01 \\
  m                 & 0.5 \\
  max\_error         & 0.000001 \\
  max\_iter          & 1000 \\
  method\_moist      & belly\_johnson \\
  method\_transport  & bagnold \\
  mixtoplayer       & True \\
  nfractions        & 11 \\
  nlayers           & 10 \\
  output\_times     & 604800.0 \\
  porosity          & 0.4 \\
  restart           & None \\
  rhoa              & 1.25 \\
  rhop              & 2650.0 \\
  rhow              & 1025.0 \\
  runup             & False \\
  scheme            & euler\_backward \\
  sigma             & 11.9 \\
  th\_bedslope      & False \\
  th\_grainsize     & True \\
  th\_humidity*     & False \\
  th\_moisture      & True \\
  th\_roughness     & True \\
  th\_salt*         & False \\
  tstart            & 0.0 \\
  tstop             & 126230400.0 \\
  z                 & 10.0 \\
  \hline
\end{longtable}

%%% Local Variables:
%%% mode: latex
%%% TeX-master: "thesis"
%%% End:


\section*{References}
\bibliographystyle{apalike} %elsarticle-harv}
\bibliography{bibliography}{}

\end{document}

%%% Local Variables:
%%% mode: latex
%%% TeX-master: t
%%% End:
