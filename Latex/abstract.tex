\chapter*{Abstract}

This thesis explores the nature of aeolian sediment availability and
its influence on aeolian sediment transport with the aim to improve
large scale and long term aeolian sediment transport estimates in
(nourished) coastal environments. The generally poor performance of
aeolian sediment transport models in coastal environments is often
accredited to limitations in sediment availability. Sediment
availability can be limited by particular properties of the bed
surface. For example, if the beach is moist or covered with
non-erodible elements, like shells. If sediment availability is
limited, the aeolian sediment transport rate is governed by the
sediment availability rather than the wind transport capacity.

Aeolian sediment availability is rather intangible as sediment
availability is not only affected by aeolian processes, but also by
marine and meteorological processes that act on a variety of spatial
and temporal scales. The Sand Motor mega nourishment provides a unique
opportunity to quantify the spatiotemporal variations in aeolian
sediment availability and its effect on aeolian sediment
transport. Aeolian sediment accumulation in the Sand Motor region is
low compared to the wind transport capacity, while the Sand Motor
itself is virtually permanently exposed to wind and accommodates large
fetches. Aeolian sediment accumulation is therefore largely determined
by the sediment availability rather than the wind transport capacity.

Multi-annual bi-monthly measurements of the Sand Motor's topography
are used for a large scale aeolian sediment budget analysis. The
analysis revealed that aeolian sediment supply from the dry beach
area, that is permanently exposed to wind, diminished a half year
after construction of the Sand Motor in 2011 due to the development of
a beach armor layer. From early 2012, two-third of the aeolian
sediment deposits originate from the intertidal beach area. The source
of aeolian sediment in the Sand Motor region is remarkable as the
intertidal beach is periodically flooded and permanently moist.

The importance of the intertidal beach area in the Sand Motor region
is tested during a six-week field campaign. Gradients in aeolian
sediment transport are measured during the field campaign as to
localize aeolian sediment source and sink areas. A consistent supply
from the intertidal beach area was measured that was temporarily
deposited at the dry beach. The temporary deposits were transported
further during high water, when sediment supply from the intertidal
beach ceased, resulting in a continuous sediment supply to the
dunes. The temporary deposition of sediment at the dry beach was
likely promoted by the presence of a berm that affects the local wind
shear. Moreover, the berm edge coincided with the onset of the beach
armor layer that might have further promoted deposition of sediment.

The measurements on spatiotemporal variations in aeolian sediment
availability and supply inspired an attempt to capture the
characteristics of aeolian sediment availability in coastal
environments in a comprehensive model approach. The resulting model
simulates spatiotemporal variations in bed surface properties and
their combined influence on aeolian sediment availability and
transport. The implementation of multi-fraction aeolian sediment
transport in the model introduces the recurrence relation between
aeolian sediment availability and transport through self-grading of
sediment.

The model was applied in a four-year hindcast of the Sand Motor mega
nourishment as first attempt to field validation. The model reproduces
the multi-annual aeolian sediment erosion and deposition volumes, and
the relative importance of the intertidal beach area as source of
aeolian sediment well. Seasonal variations in aeolian sediment
transport are incidentally missed by the model. The model accuracy is
reflected in a $\mathrm{R^2}$ value of 0.93 when comparing time series
of measured and modeled total aeolian sediment transport volumes in
the four years since construction of the Sand Motor. The results
suggest that indeed significant limitations in sediment availability,
due to soil moisture content and beach armoring, govern aeolian
sediment transport in the Sand Motor region. A comparison with a
simulation without limitation in sediment availability suggests that
aeolian sediment availability in the Sand Motor region is limited to
about 25\% of the wind transport capacity. Moreover, both spatial and
temporal variations in aeolian sediment availability as well as the
recurrence relation between aeolian sediment availability and
transport are essential to accurate long term and large scale aeolian
sediment transport estimates.


%%% Local Variables:
%%% mode: latex
%%% TeX-master: "thesis"
%%% End:
