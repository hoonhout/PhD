\chapter*{Definitions}
\label{ch:definitions}

\begin{description}
\item[wind transport capacity] [$\mathrm{kg/m/s}$] Transport capacity
  of the wind over an idealized bed. The wind transport capacity is an
  upper limit of the (sediment) transport capacity that includes the
  influence of bed surface properties.
\item[(sediment) transport capacity] [$\mathrm{kg/m/s}$] Transport
  capacity of the wind over a given bed. The (sediment) transport
  capacity accounts for the impact velocity threshold. The (sediment)
  transport capacity is an upper limit of the actual sediment
  transport.
\item[equilibrium sediment transport] Sediment transport capacity.
\item[saturated sediment transport] Sediment transport capacity.
\item[velocity threshold] [$\mathrm{kg/m/s}$] Impact velocity
  threshold at which sediment transport is sustained over a given
  bed. The threshold depends on bed surface properties that may hamper
  saltation, e.g. roughness, moist, salt, and represents the
  difference between the wind and (sediment) transport capacity.
\item[sediment availability] [$\mathrm{kg/m^2}$] Sediment currently
  available for entrainment \citep[following ][]{Kocurek1999}. The
  sediment availability includes the fluid velocity threshold at which
  sediment transport is initiated. Sediment availability may result in
  sediment supply if wind is sufficient.
\item[sediment entrainment] [$\mathrm{kg/m^2/s}$] Entrainment of
  currently available sediment by the wind and contributing to the
  sediment supply.
\item[sediment supply] [$\mathrm{kg/m/s}$] Transport of entrained
  sediment from one location to another, e.g. from marine sources to
  intertidal beach or from intertidal beach to dunes.
\item[transport-limited] Transport is determined by the wind transport
  capacity. An increase in wind speed will result in an increase in
  sediment transport as long as sediment is still available. If
  insufficient sediment is available, the coastal system becomes
  availability-limited.
\item[availability-limited] Transport is determined by the
  availability of aeolian sediment. An increase in wind speed will not
  result in an increase in sediment transport as no additional
  sediment is available. A decrease in wind speed can result in a
  transport-limited coastal system as the sediment availability might
  be able to fulfill the demand from the reduced wind.
\item[supply-limited] Availability-limited.
\item[fetch-limited] Transport is determined by the available fetch
  and therefore a wider beach or more oblique wind will result in an
  increase in sediment transport. In this thesis fetch is only
  considered a limiting factor on an idealized bed with maximum
  sediment availability (i.e. flat, dry, loose and homogeneous). The
  coastal system is considered fetch-limited if and only if the
  available fetch is shorter than the fetch necessary for the
  development of a saturated saltation cascade in these idealized
  conditions. In all other cases where the available fetch influences
  the sediment transport, the coastal system is considered
  availability-limited.
\item[sediment sorting] Spatial sorting of (sandy) sediment, either
  horizontally or vertically, due to differences in (sediment)
  transport capacity between sediment fractions.
\item[beach armoring] Emergence of non-erodible roughness elements
  from the bed that shelter (sandy) sediment from wind erosion,
  resulting in spatiotemporal differences in sediment availability.
\end{description}

%%% TeX-master: "thesis"
%%% Local Variables: %%% mode: latex %%% TeX-master: "thesis" %%% End:
