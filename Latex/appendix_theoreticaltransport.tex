\ifthenelse{\boolean{ispaper}}{
  \section{Theoretical Sediment Transport Volumes}
}{
  \chapter{Theoretical Sediment Transport Volumes}
}\label{apx:theoretical_transport}

The cumulative theoretical sediment transport volume $Q$
[$\mathrm{m^3}$] in the Sand Motor domain between September 1, 2011
and September 1, 2015 is estimated from hourly averaged measured wind
speed $u_{10}$ [m/s] and direction $\theta_u$ [$^{\circ}$] measured at
10 m height by the KNMI meteorological station in Hoek van Holland
(Figure \ref{fig:windwaves}). The wind time series are used in
conjunction with the formulation of \citet{Bagnold1937a} to obtain the
instantaneous theoretical sediment transport rate $q$ [kg/m/s]
following:

\begin{equation}
  q = C \frac{\rho_{\mathrm{a}}}{g} \sqrt{\frac{d_{\mathrm{n}}}{D_{\mathrm{n}}}} \left(u_{\mathrm{*}} - u_{\mathrm{* th}} \right)^3
\end{equation}

\noindent with the shear velocity
$u_{\mathrm{*}} = \alpha \cdot u_{10}$ m/s, the shear velocity
threshold $u_{\mathrm{* th}} = \alpha \cdot 3.87$ m/s, the conversion
factor from free-flow wind velocity to shear velocity
$\alpha = 0.058$, the air density $\rho_{\mathrm{a}}$ = 1.25
$\mathrm{kg/m^3}$, the particle density $\rho_{\mathrm{p}}$ = 2650.0
$\mathrm{kg/m^3}$, the gravitational constant $g$ = 9.81
$\mathrm{m/s^2}$, the nominal grain size $d_{\mathrm{n}}$ = 335
$\mu \mathrm{m}$ and a reference grain size $D_{\mathrm{n}}$ = 250
$\mu \mathrm{m}$.

The cumulative theoretical sediment transport volumes in onshore
($Q_{\mathrm{os}}$ [$\mathrm{m^3}$]) and alongshore ($Q_{\mathrm{as}}$
[$\mathrm{m^3}$]) direction are computed by time integration and
conversion from mass to volume following:

\begin{equation}
  \label{eq:apx_theoretical_transport}
  \begin{array}{lclcl}
    Q_{\mathrm{os}} &=& \sum q \cdot \frac{\Delta t \cdot \Delta y}{(1 - p) \cdot \rho_{\mathrm{p}}} \cdot f_{\theta_u,\mathrm{os}} &=& 110 \cdot 10^4 ~ \mathrm{m^3} \\
    Q_{\mathrm{as}} &=& \sum q \cdot \frac{\Delta t \cdot \Delta x}{(1 - p) \cdot \rho_{\mathrm{p}}} \cdot f_{\theta_u,\mathrm{as}} &=& 3 \cdot 10^4 ~ \mathrm{m^3} \\
  \end{array}
\end{equation}

\noindent where the temporal resolution $\Delta t$ = 1 h, the
alongshore span of the measurement domain $\Delta y$ = 4 km, the
approximate lateral beach width $\Delta x$ = 100 m, the porosity $p$ =
0.4 and $f_{\theta_u,\mathrm{os}}$ and $f_{\theta_u,\mathrm{as}}$ are
factors to account for respectively the onshore and alongshore wind
directions only, defined as:

\begin{equation}
  \begin{array}{lcl}
    f_{\theta_u,\mathrm{os}} &=& \max \left( 0 \quad ; \quad \cos \left( 312\,^{\circ} - \theta_u \right) \right) \\
    f_{\theta_u,\mathrm{as}} &=& \sin \left( 312\,^{\circ} - \theta_u \right) \\
  \end{array}
\end{equation}

\noindent where $\theta_u$ [$^{\circ}$] is the hourly averaged wind
direction and $312\,^{\circ}$ accounts for orientation of the original
coastline.

Note that the difference between the onshore and alongshore cumulative
theoretical sediment transport volumes (Equation
\ref{eq:apx_theoretical_transport}) of a factor 40 is determined
solely by the difference between the onshore and alongshore
cross-sections of 4 km and 100 m respectively. The sediment transport
volumes per meter width in onshore and alongshore direction are of the
same order of magnitude (275 $\mathrm{m^3/m}$ and 267 $\mathrm{m^3/m}$
respectively).

%%% Local Variables:
%%% mode: latex
%%% TeX-master: "./thesis"
%%% End:
