299,301c299,303
< cells and in vertical bed layers (2DV). Moreover, the grain size
< distribution is discretized into fractions. This allows the grain size
< distribition to vary both horizontally and vertically.
---
> cells and in vertical bed layers (2DV) that move with the
> bed. Moreover, the grain size distribution is discretized into
> fractions. This allows the grain size distribition, sediment
> availability and sediment supply to vary both horizontally and
> vertically.
364a367,373
> Waves cause stirring of the bed and mixing of the sediment that is
> available in the vertical bed layers. Therefore, wave action can cause
> finer sediment in deeper bed layers to surface thereby increasing
> local sediment availability. However, also in case of wave action
> sediment availability is limited as the model does not include a
> marine sediment source.
> 
382a392,403
> Due to the implementation of multi-fraction sediment transport
> (Equations \ref{eq:advection} and \ref{eq:erodep}), fine sediment is
> eroded from the bed surface more easily then coarse
> sediment. Consequently, the bed surface coarsens over time. The
> implementation of the sheltering effect of roughness elements
> (Equation \ref{eq:raupach}) enhances this effect as also roughness
> elements surface over time and limit sediment availability in erosive
> zones. Currently, the only counteracting process implemented in the
> model is wave stirring. Other processes, like weathering, abrasion,
> breakage or mobilization of larger shell fragments are not included,
> but tend to act on a larger temporal scale or occur irregularly.
> 
512,514c533,534
< van Holland (Figure \ref{fig:boundaryconditions}a,b). Hourly offshore
< water levels and wave heights are obtained from the Europlatform for
< the same period (Figure \ref{fig:boundaryconditions}c,d).
---
> van Holland (Figure \ref{fig:boundaryconditions}a,b) are uniformly
> imposed to the model.
585,586c605,608
< Water levels and wave heights are initially uniformly imposed to the
< model. Consequently, the tidal range, mean water level and wave
---
> Hourly offshore water levels and wave heights are obtained from the
> Europlatform for the same period (Figure
> \ref{fig:boundaryconditions}c,d).  and are initially uniformly imposed
> to the model. Consequently, the tidal range, mean water level and wave
710,716c732,741
< with net aeolian deposition (lagoon, dune lake and dunes). The zonation
< is based on the 0, 3 and 5 m+MSL contour lines that roughly correspond
< with the mean water level, maximum runup level or berm edge and the
< dune foot respectively. The average $\mathrm{R^2}$ value of the time
< series for erosion and deposition is used as benchmark. The
< $\mathrm{R^2}$ value represents the fraction of explained variance and
< is defined as:
---
> with net aeolian deposition (lagoon, dune lake and dunes). The
> zonation is based on the 0, 3 and 5 m+MSL contour lines that roughly
> correspond with the mean water level, maximum runup level or berm edge
> and the dune foot respectively. Note that zonation is strictly used as
> method to detail the spatial varying performance of the model. All
> modeled processes are identical throughout the computational
> domain. Zonation is not imposed to the model other than, indirectly,
> through topology. The average $\mathrm{R^2}$ value of the time series
> for erosion and deposition is used as benchmark. The $\mathrm{R^2}$
> value represents the fraction of explained variance and is defined as:
973,975c998,1007
< by initial shell fraction of 5\%, which might be
< overestimated. Alternatively, the initially uniform distribution of
< shells in the bed is not an accurate representation of reality.
---
> by initial shell fraction of 5\%, which might be overestimated, or the
> initially uniform distribution of shells in the bed is not an accurate
> representation of reality. Alternatively, processes that are not
> included in the model, like weathering, abrasion, breakage or
> mobilization of larger shell fragments, might explain the difference
> between model results and measurements. As weathering, abrasian and
> breakage affect the beach armoring on a larger temporal scale,
> mobilization of shells and shell fragments seems to be a more likely
> explanation to the discrepancies in the first months of the
> simulation.
1180c1212
< %%% TeX-master: "paper_hindcast_ce"
---
> %%% TeX-master: "paper_hindcast_ce_v2"
