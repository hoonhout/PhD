\section{Motivation}

Aeolian sediment transport is a prerequisite to growth and resilience
of coastal dunes. Coastal dunes function as a natural protection
against flooding from the sea. As human societies are particularly
attracted to low-lying areas near the sea, the reliability and
resilience of the protective coastal dune systems becomes vital for
economic activities and human well-being. This societal demand for a
safe and comfortable living space, that initiated the discipline of
coastal engineering, developed our understanding of coastal safety
tremendously in the past decades. The increased understanding of our
coastal systems resulted in structural mitigation of coastal risks
using rigid solutions or local nourishments \citep{Hamm2002} and the
engineering of entire coastlines worldwide \citep{Donchyts2016}.

With the increased confidence in our ability to mitigate coastal
risks, additional demands and functions for coastal flood protections
arose. Soft engineering solutions with limited environmental and
ecological impact gained preference over rigid solutions. Recently,
the exponent of soft engineering emerged as nature-based coastal flood
protections \citep{Waterman2010, deVriend2015}. Nature-based flood
protections pursue the idea of stimulating natural processes with the
aim of increasing coastal safety and is based on the assumption that
the incidental or concentrated interventions necessary for the
stimulation of nature are less intrusive than classic solutions to
coastal safety. Moreover, nature-based solutions tend to accommodate
long-term monitoring and periodic adaptation and intervention that
increases flexibility with respect to planning and execution as well
as the occurrence of coastal hazards. The increased flexibility can
make nature-based flood protection also cost-effective
\citep{vanSlobbe2013}.

An innovative example of a nature-based solution to coastal safety is
the Sand Motor \citep[or Sand Engine,][]{Stive2013}. The Sand Motor is
an artificial sandy peninsula that was constructed along the Dutch
coast in 2011. The Sand Motor provides a 21 $\mathrm{Mm^3}$ sediment
source to the Dutch coast that is to be dispersed by natural
processes, like tides and waves, over a period of about two
decades. Although the construction of the Sand Motor clearly disturbs
the coastal system, the disturbance is incidental and concentrated. In
addition, the presence of the Sand Motor theoretically decreases the
necessity of measures to mitigate coastal risks at other locations
along the Dutch coast.

The Sand Motor is the provisional pinnacle of the evolution of soft
engineering solutions to coastal safety in The Netherlands. Soft
engineering solutions started with the dynamic preservation act of
1990 that prescribes extensive nourishment program initiated to
protect The Netherlands from flooding from the sea
\citep{MinVW1990b}. Since the start of the program the distance
between nourishments and dunes increased steadily. The initial dune
and beach nourishments were replaced by foreshore nourishments as
these are more cost-effective and less intrusive to the environmental
and recreational functions of the coastal dune system. Nature-based
solutions, like the Sand Motor, typically place nourishments
kilometers away from the dune system that needs to be enforced.

With the increasing distance between nourishments and dunes, the
effectiveness of nourishments in mitigating coastal risks becomes more
difficult to assess. Ultimately the reliability of coastal dune
systems is related to the sediment volume that is contained by the
system. However, also the location in the coastal profile where the
sediment resides is important. Sediment in the dunes provides a direct
buffer against flooding in case of storm erosion, while sediment on
the beach and foreshore influences coastal safety indirectly by
depth-induced breaking of waves and consequently a reduction of the
critical dune volume required to withstand a normative storm
\citep{Walstra2016}. The sediment volume that resides in the dunes
provides arguably a more persistent protection against flooding as the
volume is typically only affected by severe storms. In contrast, the
sediment volume that resides on the foreshore and beach is affected by
seasonal nearshore bar cycles and mild storms, which increase the
uncertainty of its contribution to coastal safety. It is therefore
relevant to understand how sediment arrives in the dunes and provide a
persistent contribution to coastal safety.

A key issue is to understand sediment transport pathways from
nourishment to dunes. Many studies and sophisticated numerical models
are available that describe hydrodynamic sediment transport. However,
only a small fraction of the sediment moved in the nearshore
ultimately arrives in the dunes \citep{Aagaard2004}. It is this small
wind-induced sediment flux that provides us with the natural and
persistent coastal flood protection that nature-based solutions aim
for. In addition, this small wind-induced sediment flux gives coastal
dune systems the natural resilience to storm impacts and the
conditions for survival of persistent dune vegetation that strengthens
the coastal dune systems, like marram grass \citep{Borsje2011}. It is
also this small wind-induced sediment flux that is least understood
and consistently overestimated by existing sediment transport models.

Aeolian sediment transport models describe the wind-induced sediment
transport rate. In coastal environments these models tend to
overestimate the aeolian sediment accumulation volumes, which is often
accredited to limitations in sediment availability \citep{Houser2009,
  DelgadoFernandez2012, deVries2014b}. Sediment availability can be
limited by particular properties of the bed surface. For example, if
the beach is moist or covered with non-erodible elements, like shells
\citep{Wiggs2004, Edwards2009, Namikas2010, McKennaNeuman2012}. If
sediment availability is limited, the aeolian sediment transport rate
is governed by the sediment availability rather than the wind
transport capacity, which violates the common assumption in aeolian
sediment transport models.

This thesis explores the nature of aeolian sediment availability and
its influence on aeolian sediment transport with the aim to improve
large scale and long term aeolian sediment transport estimates in
nourished coastal environments. This work is performed within the
framework of \emph{ERC-Advanced Grant 291206 -- Nearshore Monitoring
  and Modeling (NEMO)} that aims at an integrated modeling strategy
for large scale and long term coastal sediment transport that extends
from foreshore to backshore. Improving aeolian sediment transport
estimates helps the completion of the sediment transport pathways from
foreshore to backshore and from nourishment to dunes and thereby the
assessment of measures that attempt to mitigate coastal risks,
including nature-based coastal flood protections, on their
effectiveness.

%As much of the Sand Motor's behavior is difficult to
%predict with our current understanding of coastal systems, it should
%be seen as a large scale experiment and a unique opportunity to
%explore the characteristics of nature-based flood protections.

%The path of the righteous man is beset on all sides by the iniquities
%of the selfish and the tyranny of evil men. Blessed is he who in the
%name of charity and good will shepherds the weak through the valley of
%darkness, for he is truly his brother's keeper and the finder of lost
%children. And I will strike down upon thee with great vengeance and
%furious anger those who attempt to poison and destroy my brothers. And
%you will know my name is the Lord when I lay my vengeance upon thee.

\section{Research objectives}

This thesis pursues four main research objectives. Each chapter is
dedicated to one research objective. The research objectives are
elaborated in research questions that are addressed in the concluding
chapter of this thesis (Chapter \ref{ch:conclusions}). The research
objectives and questions are formulated as:

\begin{description}
\item[Research objective \#1] Identify the main sources for aeolian
  sediment in coastal environments and particularly at the Sand Motor
  mega nourishment (Chapter \ref{ch:largescale}).

  \medskip

  The research questions related to objective \#1 are:

  \begin{enumerate}[{1.}1]
%  \item \label{q:1.1} How can aeolian sediment fluxes be
%    separated from hydrodynamic sediment fluxes at the Sand Motor mega
%    nourishment?
  \item \label{q:1.2} What is the total aeolian sediment
    supply at the Sand Motor mega nourishment?
  \item \label{q:1.3} What are the main deposition areas of
    aeolian sediment at the Sand Motor mega nourishment?
  \item \label{q:1.4} What are the main source areas of
    aeolian sediment at the Sand Motor mega nourishment?
  \item \label{q:1.5} What bed surface characteristics can
    explain any spatial variations in aeolian sediment supply at the
    Sand Motor mega nourishment?
  \item \label{q:1.6} What is the relevance of these bed
    surface characteristics for coastal systems in general?
  \item \label{q:1.7} What characteristics of a coastal system
    determine aeolian sediment supply and dune growth?
  \end{enumerate}

  \bigskip

\item[Research objective \#2] Identify the main processes that govern
  aeolian sediment availability and supply in coastal environments and
  particularly at the Sand Motor mega nourishment (Chapter
  \ref{ch:smallscale}).

  \medskip

  The research questions related to objective \#2 are:

  \begin{enumerate}[{2.}1]
%  \item \label{q:2.1} How can aeolian sediment supply be measured in
%    the field?
  \item \label{q:2.2} What bed surface characteristics are related to
    aeolian sediment supply?
  \item \label{q:2.3} What processes govern the supply of aeolian
    sediment from the source areas?
  \item \label{q:2.4} What processes govern the deposition of aeolian
    sediment in the deposition areas?
  \end{enumerate}

  \bigskip

\item[Research objective \#3] Develop a numerical model approach to
  describe the influence of spatiotemporal variations in aeolian
  sediment availability on aeolian sediment transport and harmonize
  existing model approaches to aeolian sediment availability where
  possible (Chapter \ref{ch:model}).

  \medskip

  The research questions related to objective \#3 are:

  \begin{enumerate}[{3.}1]
  \item \label{q:3.1} What are existing model approaches to describe
    the influence of aeolian sediment availability on aeolian sediment
    transport, what are the similarities and differences among them
    and which approaches are mutually exclusive?
  \item \label{q:3.2} What processes that were identified to be
    relevant to aeolian sediment availability are not covered with
    sufficient accuracy by existing model approaches?
  \item \label{q:3.3} What are the requirements for a model approach
    that harmonizes existing, mutual inclusive model approaches and is
    conceptually able to describe all processes relevant to aeolian
    sediment availability and transport?
  \end{enumerate}

  \bigskip

\item[Research objective \#4] Validate the numerical model approach to
  reproduce the location and size of sources for aeolian sediment in
  coastal environments and particularly at the Sand Motor mega
  nourishment (Chapter \ref{ch:hindcast}).

  \medskip

  The research questions related to objective \#4 are:

  \begin{enumerate}[{4.}1]
  \item \label{q:4.1} Can the calibrated numerical model reproduce the
    total aeolian sediment supply at the Sand Motor mega nourishment
    with any statistical significance?
  \item \label{q:4.2} Can the calibrated numerical model reproduce the
    main source and deposition areas at the Sand Motor mega
    nourishment?
  %\item \label{q:4.3} Has the calibrated numerical model predictive
  %  capabilities?
  \item \label{q:4.4} What implemented processes are in retrospect
    significant to the model result?
  \end{enumerate}
\end{description}

\bigskip

\section{Thesis outline}

This thesis constitutes four parts:

\begin{enumerate}[{Part} I]
\item presents field data dedicated to the aeolian sediment supply and
  transport at the Sand Motor mega nourishment.

  Chapter \ref{ch:largescale} presents a large scale aeolian sediment
  budget analysis that identifies the main suppliers of aeolian
  sediment in the Sand Motor region.

  The large scale sediment budget analysis inspired the six-week field
  campaign presented in Chapter \ref{ch:smallscale}. Gradients in
  aeolian sediment transport were measured during the field
  campaign. Gradients in aeolian sediment transport reveal areas with
  net erosion and thereby the sources of aeolian sediment. The
  measurements therefore enable a detailed analysis of processes
  governing the spatiotemporal variations in aeolian sediment
  availability as identified in the aeolian sediment budget analysis.

\item presents a numerical model for aeolian sediment
  availability and transport that is inspired by the field
  observations.

  The field data show that significant spatial variations in aeolian
  sediment availability can exist and can affect net aeolian sediment
  transport rates. The variations in aeolian sediment availability
  coincide with changes in bed surface properties, like soil moisture
  content and beach armoring. In coastal environments these bed
  surface properties typically also vary in time. Assuming that the
  spatiotemporal variations in bed surface properties indeed influence
  the aeolian sediment availability and transport, a numerical aeolian
  sediment transport model is developed.

  Chapter \ref{ch:model} presents the model philosophy and design. The
  model focuses on the incorporation of spatiotemporal variability in
  aeolian sediment availability, which is illustrated using the
  process of beach armoring. Beach armoring occurs when roughness
  elements emerge from the bed and is a typical process that causes
  spatiotemporal variations in aeolian sediment availability. Both
  conceptual cases and wind tunnel experiments are used to illustrate
  the basic model behavior.

  Chapter \ref{ch:hindcast} describes the calibration and application
  of the model to the field data presented in Chapter
  \ref{ch:largescale} as a first attempt to field validation of the
  numerical model.

%In addition, the model is
%applied to data from the Sand Motor not presented in other chapters as
%to investigate the predictive capabilities of the model.

\item concludes this thesis by addressing the research objectives and
  questions, and a discussion on the nature of aeolian sediment
  availability and corresponding modeling strategies.

\item contains appendices with specifics on the reference model, the
  numerical model implementation and available model settings
  presented in Chapter \ref{ch:model}.

\end{enumerate}

%%% Local Variables:
%%% mode: latex
%%% TeX-master: "thesis"
%%% End:
