\begin{otherlanguage}{dutch}
\chapter*{Samenvatting}
Dit proefschrift onderzoekt de invloed van beschikbaarheid van eolisch
sediment op eolisch sediment transport met als doel het verbeteren van
grootschalige en lange termijn voorspellingen van eolisch sediment
transport in (gesuppleerde) kustgebieden. Bestaande modellen voor
eolisch sediment transport presteren in het algemeen slecht in
kustgebieden. De slechte prestaties worden dikwijls geweten aan een
beperkte beschikbaarheid van eolisch sediment. De beschikbaarheid van
eolisch sediment kan worden beperkt door specifieke eigenschappen van
het strandoppervlak. Als bijvoorbeeld het strand vochtig is of bedekt
met niet-erodeerbare elementen, zoals schelpen, kan de beschikbaarheid
van sediment worden beperkt. In dat geval wordt de eolische sediment
flux bepaald door de beschikbaarheid van sediment in plaats van de
transportcapaciteit van de wind.

De beschikbaarheid van eolisch sediment is een redelijk ongrijpbaar
fenomeen, omdat deze niet alleen wordt be{\:i}nvloed door eolische
processen, maar ook door marine en meteorologische processen die op
verschillende ruimtelijke en temporele schalen actief zijn. De
Zandmotor megasuppletie biedt een unieke gelegenheid om de temporele
en ruimtelijke variaties in de beschikbaarheid van eolisch sediment en
het effect daarvan op eolisch sediment transport te
kwantificeren. Instuifvolumes rond de Zandmotor zijn klein in
vergelijking met de transportcapaciteit van de wind, terwijl de
Zandmotor met haar grote strijklengtes zelf vrijwel permanent
blootgesteld is aan wind. De instuiving van sediment wordt daar dus
grotendeels bepaald door de beschikbaarheid van sediment in plaats van
de transportcapaciteit van de wind.

Meerjarige tweemaandelijkse metingen van de topografie van de
Zandmotor zijn gebruikt voor een grootschalige eolische sediment
budget analyse. Uit de analyse bleek dat de aanvoer van eolisch
sediment vanaf het droge strand, dat permanent blootgesteld is aan de
wind, vanaf een half jaar na de aanleg van de Zandmotor in 2011 sterk
is verminderd als gevolg van het ontstaan van een schelpenlaag. Vanaf
het begin van 2012 is tweederde van het instuifvolume afkomstig uit
het intergetijdengebied. Deze bron van eolische sediment rond de
Zandmotor is opmerkelijk omdat het intergetijdenstrand periodiek
overstroomd en permanent vochtig is.

Het belang van het intergetijdengebied strand rond de Zandmotor wordt
bevestigd tijdens een zes weken durende veld campagne. Tijdens deze
campagne zijn gradi{\:e}nten in eolisch sediment transport gemeten de
bron van eolisch sediment te bepalen. De gemeten constante aanvoer
vanuit het intergetijdengebied bleek tijdelijk te sedimenteren op het
droge strand. Deze tijdelijke afzettingen werden tijdens hoogwater
verder getransporteerd wanneer de sediment aanvoer vanaf het
intergetijdenstrand stagneerde. Hierdoor ontstond een continue toevoer
van sediment naar de duinen. De tijdelijke afzetting van sediment op
het droge strand werd vermoedelijk bevorderd door de aanwezigheid van
een berm die de lokale schuifspanning van de wind
be{\:i}nvloedt. Bovendien viel de rand van de berm samen met het begin
van de schelpenlaag die het neerslaan van sediment mogelijk verder
bevorderd heeft.

De gemeten ruimtelijke en temporele variaties in de beschikbaarheid
van eolisch sediment hebben een poging ge{\:i}nspireerd om de
kenmerken van beschikbaarheid van eolisch sediment aan de kust te
vatten in een alomvattende model aanpak. Het resulterende model
simuleert variaties in tijd en ruimte van eigenschappen van het
strandoppervlak en hun gezamenlijke invloed op de beschikbaarheid en
transport van eolisch sediment. De implementatie van eolisch sediment
transport over meerdere korrelgrootte fracties in het model
introduceert de recurrente betrekking tussen de beschikbaarheid en het
transport van eolisch sediment door middel van zelfgradering van
sediment.

Het model is toegepast in een vierjarige hindcast van de Zandmotor
megasuppletie als eerste poging tot veldvalidatie. Het model
reproduceert de meerjarige erosie en depositie volumes van eolisch
sediment, en het relatieve belang van het intergetijdengebied als bron
van eolisch sediment, goed. Seizoensafhankelijke variaties in eolisch
sediment transport worden soms gemist door het model. De
nauwkeurigheid van het model is weerspiegeld in een $\mathrm{R^2}$
waarde van 0,93 wanneer gemeten en gemodelleerde tijdseries voor het
totaal door de wind getransporteerde sedimentvolume in de vier jaar na
constructie van de Zandmotor worden vergeleken. De resultaten
suggereren dat significante beperkingen in sediment beschikbaarheid,
als gevolg van het vochtgehalte van het strand en het vormen van een
schelpenlaag, inderdaad bepalend zijn voor het eolisch sediment
transport rond de Zandmotor. Een vergelijking met een simulatie zonder
beperking in beschikbaarheid van sediment suggereert dat de
beschikbaarheid van eolisch sediment rond de Zandmotor is beperkt tot
ongeveer 25\% van de transportcapaciteit van de wind. Bovendien zijn
zowel de ruimtelijke en temporele variaties in de beschikbaarheid van
eolisch sediment, evenals de recurrente betrekking tussen de
beschikbaarheid en transport van eolisch sediment essentieel voor een
nauwkeurige lange termijn en grootschalige voorspellingen van eolisch
sediment transport.
\end{otherlanguage}


%%% Local Variables:
%%% mode: latex
%%% TeX-master: "thesis"
%%% End:
