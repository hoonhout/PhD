\begin{otherlanguage}{dutch}
\chapter*{Samenvatting}
Dit proefschrift onderzoekt de invloed van de beschikbaarheid van
eolisch sediment op het transport van eolisch sediment. Het doel van
dit onderzoek is het verbeteren van grootschalige
langetermijnvoorspellingen van eolisch sedimenttransport in
(gesuppleerde) kustgebieden. Bestaande eolisch
sedimenttransportmodellen presteren in het algemeen matig ten opzichte
van metingen in kustgebieden. De matige prestaties worden dikwijls
geweten aan een beperkte sedimentbeschikbaarheid. Specifieke
eigenschappen van het strandoppervlak, zoals de bodemvochtigheid of de
aanwezigheid van niet-erodeerbare elementen als schelpen,
be{\"i}nvloeden de sedimentbeschikbaarheid. Bij beperkte
sedimentbeschikbaarheid wordt het sedimenttransport niet meer bepaald
door de transportcapaciteit van de wind, maar door de
sedimentbeschikbaarheid.

De eolisch sedimentbeschikbaarheid is een tamelijk ongrijpbaar
fenomeen, omdat naast eolische ook marine en meteorologische processen
de sedimentbeschikbaarheid be{\"i}nvloeden. Bovendien vari{\"e}ren
deze processen op verschillende ruimtelijke en temporele schalen. De
Zandmotor, een in 2011 aangelegde megasuppletie van 21 $\mathrm{Mm^3}$
langs de Delflandse kust, is gebruikt om de temporele en ruimtelijke
variaties in de beschikbaarheid en transport van eolisch sediment te
kwantificeren. Instuifvolumes rond de Zandmotor zijn klein in
vergelijking met de transportcapaciteit van de wind, ondanks het grote
suppletievolume, de grote strijklengtes en de vrijwel permanente
blootstelling aan wind. Daarom is de sedimentbeschikbaarheid
waarschijnlijk van significante invloed op de instuifvolumes in dit
gebied.

Een grootschalige eolisch sedimentbudgetanalyse is uitgevoerd op basis
van meerjarige tweemaandelijkse topografische metingen van de
Zandmotor. De analyse toont aan dat vanaf een halfjaar na de aanleg
van de Zandmotor de eolisch sedimentaanvoer van het droge strand sterk
is verminderd. De afname is waarschijnlijk het gevolg van het ontstaan
van een schelpenlaag. In de daarop volgende jaren is tweederde van het
instuifvolume afkomstig uit de laaggelegen stranden rond de Zandmotor
die periodiek onderstromen en daarom grotendeels vochtig zijn.

Tijdens een zes weken durende veldcampagne is de sedimenttoevoer vanaf
de laaggelegen stranden rond de Zandmotor geverifieerd. Gradi{\"e}nten
in eolisch sedimenttransport zijn gemeten om de bron van eolisch
sediment te bepalen. De aanvoer vanuit het intergetijdengebied bleek
tijdelijk te sedimenteren op het hogere en droge strand. Deze
tijdelijke afzettingen werden tijdens hoogwater verder
getransporteerd, wanneer de sedimentaanvoer vanaf het
intergetijdenstrand stagneerde. Hierdoor ontstond een continue toevoer
van sediment richting de duinen. De tijdelijke afzetting van sediment
op het droge strand werd vermoedelijk bevorderd door de aanwezigheid
van een berm die de lokale schuifspanning van de wind
be{\"i}nvloedt. Bovendien viel de rand van de berm samen met het begin
van de schelpenlaag die het neerslaan van sediment mogelijk verder
bevorderd heeft.

De veldmetingen zijn de basis geweest voor de ontwikkeling van een
modelaanpak die de invloed van sedimentbeschikbaarheid op eolisch
sedimenttransport beschrijft. Het ontwikkelde model simuleert
ruimtelijke en temporele variaties in de samenstelling van het
strandoppervlak en hun gezamenlijke invloed op de beschikbaarheid en
transport van eolisch sediment. Het model onderscheidt meerdere
korrelgroottefracties waardoor een recurrente betrekking tussen de
beschikbaarheid en het transport van eolisch sediment ontstaat als
gevolg van zelfgradering van sediment.

Het model is toegepast op de Zandmotor en vergeleken met de meerjarige
topografische metingen als eerste veldvalidatie. Het model
reproduceert de meerjarige erosie en depositie volumes van eolisch
sediment, en het relatieve belang van het intergetijdengebied als bron
van eolisch sediment, goed. Seizoensafhankelijke variaties in eolisch
sedimenttransport worden soms gemist door het model. De nauwkeurigheid
van het model is weerspiegeld in een $\mathrm{R^2}$ waarde van 0,93
wanneer gemeten en gemodelleerde tijdseries voor het totaal door de
wind getransporteerde sedimentvolume in de vier jaar na constructie
van de Zandmotor worden vergeleken. De resultaten suggereren dat
significante beperkingen in sedimentbeschikbaarheid, als gevolg van
het bodemvochtgehalte en het vormen van een schelpenlaag, inderdaad
bepalend zijn voor het eolisch sedimenttransport rond de
Zandmotor. Een vergelijking met een simulatie zonder beperkingen in de
sedimentbeschikbaarheid suggereert dat de beschikbaarheid van eolisch
sediment rond de Zandmotor is beperkt tot ongeveer 25\% van de
transportcapaciteit van de wind. Bovendien zijn zowel de ruimtelijke
en temporele variaties in de sedimentbeschikbaarheid, evenals de
recurrente betrekking tussen de sedimentbeschikbaarheid en -transport
essentieel voor een nauwkeurige grootschalige langetermijnvoorspelling
van eolisch sedimenttransport.
\end{otherlanguage}


%%% Local Variables:
%%% mode: latex
%%% TeX-master: "thesis"
%%% End:
