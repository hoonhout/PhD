This thesis explored the nature of aeolian sediment availability.
Aeolian sediment availability is generally associated with the shear
velocity threshold. Alternatively, sediment availability can be
expressed in terms of critical fetch \citep{Bauer2002} or explicitly
following \citet{deVries2014b}. The latter two approaches deviate
rather radically from the legacy of aeolian research. Both discard the
abundantly available relations between bed surface properties,
sediment availability and the shear velocity threshold. Instead, they
describe sediment availability in new terms for which no
quantification is known. In Chapter \ref{ch:model} it is argued that
the new approaches have a right to exist as they allow for an
increased complexity in situations that can be described by
models. Ultimately, the approach of \citet{deVries2014b} is adopted
for this thesis and adapted to support multi-fraction sediment
transport in order to simulate sediment sorting and beach armoring
that introduces feedback between aeolian sediment availability and
transport in aeolian sediment transport modeling.

The model approach presented in this thesis is a unification of the
classic approach based on the shear velocity threshold and the
approach of \citet{deVries2014b}. Whereas \citet{deVries2014b} use a
user-defined value for the sediment availability $m_{\mathrm{a}}$ (or
$S_{\mathrm{e}}$) to truncate the instantaneous sediment transport in
their model, the approach presented in this thesis uses simulation to
determine the local sediment availability $m_{\mathrm{a,k}}$. The
weighting factor $\hat{w}_k$ is subsequently adapted to the local
sediment availability (Equation \ref{eq:erodep_multi}). Since the
weighting factor $\hat{w}_k$ is essentially a space- and
time-dependent modification to the shear velocity threshold, this new
approach connects the field of availability-limited aeolian sediment
transport modeling with the long history of aeolian
research.

Availability-limited aeolian sediment transport modeling can be
further improved by distinguishing between the fluid and impact
threshold in future versions of the presented model. The presented
model provides a framework in which such distinction can be made
rather naturally. The bed interaction parameter essentially implements
this distinction already for armored beach surfaces, but a more
generic implementation is required for coastal systems with large
spatial variations in sediment availability. A key issue is still that
relations between bed surface properties, sediment availability and
the shear velocity threshold are typically derived as bulk formulation
for the fluid and impact threshold together. A proper implementation
of the fluid and impact threshold in the presented model therefore
depends on a substantial investment in data collection
\citep{Martin2016}.

The presented model for availability-limited aeolian sediment
transport can also be applied to availability-abundant coastal
systems. However, much of the complexity in coastal aeolian sediment
transport modeling is related to aeolian sediment availability. The
approach might be considered too complex for more regular situations
where sediment availability is abundant. A key issue is that assessing
whether a coastal environment is availability-limited or not, is not
trivial. In this thesis established formulations for equilibrium
aeolian sediment transport are used to roughly estimate the sediment
availability of coastal systems. However, if the presented model is
proven to be sufficiently accurate in availability-abundant coastal
systems, it might be used to formulate rules of thumb to assess
coastal systems on their sediment availability more
accurately. Subsequently, the model might be used to assess these
coastal systems that are found to be availability-limited or
facilitate the formulation of aggregated relations between aeolian
sediment availability and transport that would serve a more rapid
assessment.

In hindsight, the chapters in this thesis reveal a certain chronology
in understanding the phenomenon of aeolian sediment availability.
Deducing the significance of sediment availability from the large
scale sediment budget analysis or the small scale sediment transport
measurements was not trivial as sediment availability appeared to be
rather intangible. This started with the slightly ambiguous use of
terminology in literature where, for example, \emph{sediment supply}
is often mistaken to be equal to \emph{sediment availability} or the
\emph{shear velocity threshold}. In addition, various aeolian, marine
and meteorological processes affect aeolian sediment availability
and/or the shear velocity threshold differently and on various
temporal and spatial scales. Consequently, significant time was spent
to narrow down the essence of sediment availability and define a
distinctive terminology accordingly. The phenomenon of aeolian
sediment availability was made tangible in a numerical model for
aeolian sediment availability and transport that is unique in that it
describes aspects essential to aeolian sediment availability and
transport modeling that have previously been subject to research, but
never been combined in a comprehensive model approach. The main
contribution of the model is the \emph{combined} simulation of:

\begin{enumerate}
\item Temporal variations in aeolian sediment availability and
  transport.
\item Spatial variations in aeolian sediment availability and
  transport.
\item Recurrence relation between aeolian sediment availability and
  transport through self-grading of sediment.
\item Simulation of multiple availability-limiting processes and their
  combined influence on aeolian sediment transport.
\item Natural differentiation between fluid and impact shear velocity
  threshold (not implemented).
\end{enumerate}

\section{Model improvements}

As illustrated by the Sand Motor hindcast presented in Chapter
\ref{ch:hindcast}, the combined influence of various aspects of
aeolian sediment availability is essential to obtain reliable
estimates of aeolian sediment transport and dune growth for which the
presented model provides a general framework. Notwithstanding that
model development just started and needs further extension,
calibration and validation to make it generally applicable.

Based on the Sand Motor hindcast several opportunities for future
model improvement have been identified:

\begin{itemize}
\item Support for local variations in wind shear due to morphological
  feedback.

  Neither deposition in front of the dunes nor deposition in the lee
  of the Sand Motor crest is currently simulated as morphological
  feedback with the wind is not taken into account. Given the
  discrepancy in the spatial distribution of aeolian sediment deposits
  between measurements and model result, it seems advisable to provide
  the model with local variations in wind shear. The model of
  \cite{Kroy2002} based on the derivation of the local wind field by
  \citet{Weng1991} might provide a description of the local variations
  in wind shear for which the computational effort relates well to the
  presented model.

%\item Support for local variations in wind shear due to the presence
%  of vegetation.
%
%  No deposition in the dunes is currently simulated as the reduction
%  in wind shear due to the presence of vegetation is not taken into
%  account. The model of \citet{Duran2013} as an extension to the model
%  of \citet{Kroy2002} might provide this functionality.

\item Support for the effect of wind gusts.

  The calibrated model is forced by an hourly averaged wind time
  series. The use of hourly averaged values for wind speed neglects
  the gustiness of the wind. Wind gusts might influence sediment
  transport significantly as the relation between wind speed and
  sediment transport is nonlinear. However, providing the model with
  high resolution wind time series would require a less diffusive
  numerical scheme that would likely not be computationally feasible
  for long-term simulations. Moreover, since saltation is not purely
  an advective mode of transport, the assumption of advection might be
  violated for very short time scales as interaction with the bed
  becomes dominant.

  As an alternative, the influence of gusts can be parameterized. It
  can be argued that some persistence is needed for gusts to influence
  sediment transport, resulting in a lower boundary of the temporal
  resolution of the wind time series. The distribution of the wind
  speed with respect to the hourly average can then provide a basis
  for a gustiness factor that increases the global wind shear.

\item Support for differentiation between the fluid and impact shear
  velocity thresholds.

  The sediment transport capacity is currently implemented identical
  for the initiation and continuation of motion as no distinction
  between the fluid and impact threshold is made. The Sand Motor
  hindcast illustrates how this restriction affects the influence of
  roughness elements on aeolian sediment availability. Similar to the
  implementation of the bed interaction parameter, that distinguishes
  between the grain size distribution in the bed and the air, a
  distinction between fluid and impact threshold can be implemented.

  The right-hand side of the advection equation (Equation
  \ref{eq:erodep_multi}) can be modified according to:

  \begin{equation}
    \label{eq:erodep_split}
    E_k - D_k = \min \left( 
      \frac{\partial m_{\mathrm{a},k}}{\partial t} \quad ; \quad 
      \frac{\hat{w}_k}{T} \cdot \left[
        (1 - S_k) \cdot c^{\mathrm{fluid}}_{\mathrm{sat},k} +
        S_k \cdot c^{\mathrm{impact}}_{\mathrm{sat},k} - c_k
      \right]
    \right)
  \end{equation}

  \noindent where $c^{\mathrm{fluid}}_{\mathrm{sat},k}$
  [$\mathrm{kg/m^2}$] and $c^{\mathrm{fluid}}_{\mathrm{sat},k}$
  [$\mathrm{kg/m^2}$] are the sediment transport capacity associated
  with the fluid and impact threshold respectively and $S_k$ [-] is
  the degree of saturation defined as:

  \begin{equation}
    S_k = \sum_{k=1}^{n_{\mathrm{k}}} \frac{c_k}{c^{\mathrm{impact}}_{\mathrm{sat},k}}
  \end{equation}

  \noindent Note that the weighting factor $\hat{w}_k$ already
  distinguishes between sediment in the air and in the bed, depending
  on the saturation and the bed interaction parameter and therefore
  appears outside the brackets.

  Although the presented model conceptually allows to differentiate
  between the impact and fluid threshold, empirical data to quantify
  the differentiation is lacking. This model improvement is therefore
  hypothetical at the current stage of development.

\item Support for independent definition of the active bed layer.

  The top bed composition layer currently acts as active bed layer,
  but at the same time defines the vertical resolution of the sorting
  and armoring processes. As these are two fundamentally different
  properties of the model, it is advisable to define the active bed
  layer separate from the numerical resolution. A probability
  distribution can be defined that describes the probability of
  sediment to be eroded from a specific layer, which would logically
  decrease with the depth. The bed composition layer thickness would
  than uniquely determine the vertical resolution of sorting and
  armoring.

\item Use of online coupling with other models

  The Sand Motor hindcast showed the importance of an accurate
  description of the hydrodynamics for accurate estimates of the
  development of the aeolian sedimentation and erosion
  volumes. Similarly, groundwater seepage might influence the aeolian
  sediment deposition around the lagoon and dune lake, which would
  require a description of the groundwater level to be implemented.

  It can be questioned if such detailed descriptions of hydrodynamic
  and hydrological processes are still within the scope of an aeolian
  sediment transport model. Alternatively, online model coupling with
  dedicated models for near-shore hydrodynamics and hydrology can be
  pursued. A Basic Model Interface (BMI) was implemented to
  accommodate model coupling.

\end{itemize}

\section{Model validation}

The Sand Motor appeared to be a valuable field site for investigating
the influence of aeolian sediment availability on long-term aeolian
sediment transport. The construction height, dry surface area and
considerable beach armoring and compartmentalization amplify the
processes governing aeolian sediment availability. These
characteristics make the Sand Motor also a peculiar coastal site. The
relative importance of processes governing aeolian sediment
availability are likely to be different at a more ordinary coastal
site. For example, beach armoring will not be as important on narrow
beaches that are frequently flooded. Preliminary model results for
more ordinary coastal sites suggest that the extent of the
compartmentalization of the beach is less, but spatiotemporal
variations in aeolian sediment availability still influence aeolian
sediment transport significantly. A comparison between field
measurements from more ordinary coastal sites and model results can
provide insight in the relative importance of these processes in
general.

%Aeolian sediment availability is generally associated with the shear
%velocity threshold. Based on the subaerial evolution of the Sand Motor
%mega nourishment it is illustrated that practical application of the
%shear velocity threshold to describe the local aeolian sediment
%availability is not trivial. The shear velocity threshold is used to
%describe both the change in sediment transport capacity, associated
%with the impact threshold, as well as the change in sediment
%availability, associated with the fluid threshold. Moreover, the shear
%velocity is influenced by spatiotemporal variations in bed surface
%properties and the feedback between sediment transport and sediment
%availability, that require a process-based description of the shear
%velocity threshold.
%
%In Chapter \ref{ch:model} different approaches 
%
%
%the only existing model
%approach that could potentially fulfill these requirements is the
%availability-limited advection scheme following
%\citet{deVries2014b}. This approach is extended with a process-based
%description of soil moisture and sediment sorting and beach
%armoring. What is not obvious from the model description in Chapter
%\ref{ch:model} is that this approach appears to be equivalent to the
%traditional approach based on the shear velocity threshold. However,
%the shear velocity threshold is captured in an advection framework
%that allows for feedback with between sediment availability and
%transport as well as differentiation between the fluid and impact
%threshold.





% and
%reflects a certain chronology in understanding the phenomenon.  In
%Chapter \ref{ch:largescale} sediment availability is discussed in
%terms of potential and actual sedimentation volumes at the Sand Motor
%mega nourishment. The difference is generally associated with aeolian
%sediment availability and expressed in a shear velocity
%threshold. Alternatively, sediment availability can be expressed in
%terms of critical fetch \citep{Bauer2002} or explicitly following
%\citet{deVries2014b}. In Chapter \ref{ch:model} it is argued that
%these approaches are related, but differ in the complexity of
%situations they can describe. The approach of \citet{deVries2014b} is
%adopted and extended with a process-based description of soil moisture
%and sediment sorting and beach armoring. In Chapter \ref{ch:hindcat}
%this new approach is applied to the Sand Motor mega nourishment, which suggests that...




%Observations at the Sand Motor mega nourishment
%inspired the development of a numerical model that focuses on the
%quantification of aeolian sediment availability. Therefore the
%\textsc{AeoLiS} model should, in its current stage, be seen as a
%sediment availability rather than a sediment transport model.
%
%Past research on aeolian sediment availability focused mainly on the
%use of shear velocity thresholds. 

% the nature of limitations in sediment availability, the paradox, the connections


% time and space variability introduced, but initiation and conitunation distinction still to do

% ultimately: simple model








% average wind vs gusts


% bed composition layers


% winter vs. summer fluxes

% offshore fluxes

% alongshore variability

% salt crusts and other supply limiters

%%% Local Variables:
%%% mode: latex
%%% TeX-master: "thesis"
%%% End:
