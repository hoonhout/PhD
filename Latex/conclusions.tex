This thesis concludes with addressing the research questions
formulated in the introductory chapter (Chapter
\ref{ch:introduction}). Each question is addressed with relevant
references to the four main chapters \ref{ch:largescale} to
\ref{ch:hindcast}.

\begin{description}
\item[Research objective \#1] Identify the main sources for aeolian
  sediment in coastal environments and particularly at the Sand Motor
  mega nourishment.

  \medskip

  The research questions and answers related to objective \#1 are:

  \begin{enumerate}[{1.}1]
  \item What is the total aeolian sediment supply at the Sand Motor
    mega nourishment?

    The total aeolian sediment supply accumulated to 400.000
    $\mathrm{m^3}$ in the first four years after construction of the
    Sand Motor\rrq{1.2ls}. About 120.000 $\mathrm{m^3}$ accumulated in
    the first half year after construction. From January 2012 the
    accumulation rate of aeolian sediment reduced with two-third to
    about 80.000 $\mathrm{m^3/yr}$.

  \item What are the main deposition areas of aeolian sediment at the
    Sand Motor mega nourishment?

    Aeolian sediment in the Sand Motor region is deposited in the
    dunes (50\%), dune lake (25\%) and lagoon\rrq[25\%, ]{1.3ls}. The
    deposits in the dunes increased with respect to the dune lake and
    lagoon over the course of four years since construction of the
    Sand Motor. In addition, aeolian sediment is likely to be
    deposited offshore as well. The associated sediment volume is
    unknown, but estimated by the numerical model as 10\% of the
    measured deposition volume\rrq{1.3hc}.

  \item What are the main source areas of aeolian sediment at the Sand
    Motor mega nourishment?

    Aeolian sediment in the Sand Motor region originates from the dry
    beach area (aeolian zone, 33\%) and the intertidal and low-lying
    supratidal beach areas\rrq[mixed zones, 67\%, ]{1.4ls}. The
    relative importance of the mixed zones is notable as it is
    periodically flooded and the majority of the northern mixed zone
    is oriented unfavorable with respect to the wind.
  
  \item What bed surface characteristics can explain any spatial
    variations in aeolian sediment supply at the Sand Motor mega
    nourishment?

    Aeolian sediment supply from the Sand Motor's dry beach area
    (aeolian zone) is likely to be reduced due to beach
    armoring. However, beach armoring is probably not the only
    relevant bed surface characteristic as the formation of salt
    crusts has a similar limiting effect on aeolian sediment
    supply\rrq{1.5ls}.

    The reduction of aeolian sediment supply from the aeolian zone
    makes the intertidal and low-lying supratidal beach areas (mixed
    zones) relatively important as supplier of aeolian sediment. The
    periodic flooding and high moisture contents are known to limit
    aeolian sediment supply, but apparently to a lesser extent than
    beach armoring\rrq{1.5ls}.

  \item What is the relevance of these bed surface characteristics for
    coastal systems in general?

    While high moisture contents are found at any tidal beach, beach
    armoring is especially relevant to large scale nourishments
    constructed above storm surge level.%\rrq{1.6ls}
    Sand borrowed from
    the sea bottom and used to widen or strengthen the coast tends to
    contain many shells and other roughness elements. As these
    elements tend to emerge from the bed due to winnowing of fine
    sediment, beach armoring is an inherent process on such
    beaches. In case beach armoring occurs, aeolian sediment supply
    from the intertidal and low-lying supratidal beach areas (mixed
    zones) becomes more important and consequently also soil moisture
    content, infiltration and evaporation rates.

  \item What characteristics of a coastal system determine aeolian
    sediment supply and dune growth?

    The intertidal and low-lying supratidal beach areas seem to be a
    more reliable indicator for aeolian sediment supply and dune
    growth along coasts with beach armoring than (dry) beach
    fetch\rrq{1.7ss}.
  
  \end{enumerate}

  \bigskip

\item[Research objective \#2] Identify the main processes that govern
  aeolian sediment availability and supply in coastal environments and
  particularly at the Sand Motor mega nourishment.

  \medskip

  The research questions and answers related to objective \#2 are:

  \begin{enumerate}[{2.}1]
%  \item How can aeolian sediment supply be measured in the field?
%
%  Aeolian sediment supply can be determined by integration of spatial
%  gradients in aeolian sediment transport. Aeolian sediment transport
%  gradients can be measured in the field by an array of saltation
%  measurement devices oriented in downwind direction\rrq{2.1ls}.

  \item What bed surface characteristics are related to aeolian sediment
    supply?

    Significant changes in spatial gradients in aeolian sediment
    transport at the Sand Motor coincide with the presence of a beach
    armor layer. This suggests that beach armoring is a dominant
    process in the reduction of aeolian sediment
    supply\rrq{2.2ss}. Spatial gradients in aeolian sediment transport
    also seem to be related to topographic features, like the
    transition from berm slope to berm flat. Such features seem to
    promote deposition of aeolian sediment (negative supply).
  
  \item What processes govern the supply of aeolian sediment from the
    source areas?

    Aeolian sediment supply at the Sand Motor is governed by the
    development of a beach armor layer\rrq{2.3ss1}. The reduction of
    aeolian sediment supply from the dry beach area increases the
    importance of aeolian sediment supply from the intertidal beach
    areas. Aeolian sediment supply from the mixed zone is governed by
    the periodic flooding of the intertidal beach. The timescales
    involved in the flooding and drying of the intertidal beach seems
    to be short, resulting in an swift response of the sediment supply
    to the instantaneous waterline position. Local deposits on the
    berm flat seem to act as temporary sediment source during high
    water or high moisture contents resulting in a continuous supply
    from the intertidal beach\rrq{2.3ss2}.

  \item What processes govern the deposition of aeolian sediment in
    the deposition areas?

    Aeolian sediment deposits at the Sand Motor are found in typical
    areas with either low shear velocities, due to the presence of
    vegetation (dunes) or morphological feedback with the wind (lee of
    the Sand Motor crest), or high shear velocity threshold, due to
    high moisture contents or free water surfaces (dune lake, lagoon
    and offshore).  Local deposits on the berm flat seem to act as
    temporary sediment source during high water or high moisture
    contents resulting in a continuous supply from the intertidal
    beach\rrq{2.3ss2}. Alongshore variations in sediment deposition
    seem to be caused by blockage of aeolian sediment transport
    pathways\rrq{2.4ls}.

\end{enumerate}

\bigskip

\item[Research objective \#3] Develop a numerical model approach to
  describe the influence of spatiotemporal variations in aeolian
  sediment availability on aeolian sediment transport and harmonize
  existing model approaches to aeolian sediment availability where
  possible.

  \medskip

  The research questions and answers related to objective \#3 are:

  \begin{enumerate}[{3.}1]
  \item What are existing model approaches to describe the influence
    of aeolian sediment availability on aeolian sediment transport,
    what are the similarities and differences among them and which
    approaches are mutually exclusive?

    Three approaches can be distinguished in literature: the shear
    velocity threshold, the critical fetch and the explicit
    formulation of sediment availability\rrq{3.1m}. All approaches are
    related, but differ in the amount of spatiotemporal variability in
    sediment availability they can allow. The approach based on
    critical fetch is mutually exclusive with the approach based on
    the explicit formulation of sediment availability as the latter
    provides the critical fetch as model result. The approach based on
    an explicit formulation of sediment availability is, in the form
    presented in this thesis, a spatiotemporal advection framework for
    the shear velocity threshold that in addition allows for feedback
    between aeolian sediment availability and transport as well as
    differentiation between the impact and fluid threshold.
  
  \item What processes that were identified to be relevant to aeolian
    sediment availability are not covered with sufficient accuracy by
    existing model approaches?

    Spatiotemporal variations in beach armoring is shown to be a
    governing process at the Sand Motor mega nourishment and likely
    other nourished beaches. Especially the spatiotemporal variations
    are not sufficiently accurately described in existing models for
    aeolian sediment transport\rrq{3.2}.

  \item What are the requirements for a model approach that harmonizes
    existing, mutual inclusive model approaches and is conceptually
    able to describe all processes relevant to aeolian sediment
    availability and transport?

    The approach is based on the legacy of aeolian research, being the
    abundantly available relations between aeolian sediment
    availability and transport. It describes feedback between wind and
    transport as well as sediment availability and
    transport\rrq{3.2m}. It distinguishes between the fluid and impact
    threshold\rrq{3.2d}.

\end{enumerate}

\bigskip

\item[Research objective \#4] Validate the numerical model approach to
  reproduce the location and size of sources for aeolian sediment in
  coastal environments and particularly at the Sand Motor mega
  nourishment.

  \medskip

  The research questions and answers related to objective \#4 are:

  \begin{enumerate}[{4.}1]
  \item Can the calibrated numerical model reproduce the total aeolian
    sediment supply at the Sand Motor mega nourishment with any
    statistical significance?

    Yes. The total aeolian sediment supply over the course of 4 years
    is represented with an $\mathrm{R^2}$ value of 0.93 and an RMSE of
    $3 \cdot 10^4 ~ \mathrm{Mm^3}$\rrq{4.1}.

  \item Can the calibrated numerical model reproduce the main source
    and deposition areas at the Sand Motor mega nourishment?

    Yes. the relative contribution of the intertidal beach was
    estimated to be 55\% based on the large scale sediment budget
    analysis, which is well represented by the calibrated
    model.%\rrq{4.2}.

%\item Has the calibrated numerical model predictive capabilities?

  \item What implemented processes are in retrospect significant to
    the model result?

    Both the drying of the intertidal beach and sediment sorting and
    beach armoring are crucial for the model result. Moreover, both
    the spatial and temporal variations affect the model result
    significantly\rrq{4.3hc}.

\end{enumerate}
\end{description}

%%% Local Variables:
%%% mode: latex
%%% TeX-master: "thesis"
%%% End:
