This thesis concludes with addressing the research questions
formulated in the introductory chapter (Chapter
\ref{ch:introduction}).

\begin{description}
\item[Research objective A] Identify the main sources for aeolian
  sediment at the Sand Motor mega nourishment (Chapter
  \ref{ch:largescale}).

  \medskip

  The research questions and answers related to this objective are:

  \begin{enumerate}[{A}1]
  \item What is the total aeolian sediment supply at the Sand Motor
    mega nourishment?

    The total aeolian sediment supply accumulated to 400.000
    $\mathrm{m^3}$ in the first four years after construction of the
    Sand Motor in 2011. The average accumulation rate during these
    four years was therefore 100.000 $\mathrm{m^3/yr}$. In the first
    half year after construction the total accumulation was 120.000
    $\mathrm{m^3}$, indicating an average accumulation rate of 240.000
    $\mathrm{m^3/yr}$. From January 2012 the accumulation rate of
    aeolian sediment reduced with two-third to about 80.000
    $\mathrm{m^3/yr}$.

  \item What are the main deposition areas of aeolian sediment at the
    Sand Motor mega nourishment?

    Aeolian sediment in the Sand Motor region is deposited in the
    dunes (50\%), dune lake (25\%) and lagoon (25\%). The deposits in
    the dunes increased with respect to the dune lake and lagoon over
    the course of four years since construction of the Sand Motor. In
    addition, aeolian sediment is likely to be deposited offshore as
    well. The associated sediment volume is unknown, but estimated by
    the numerical model as 10\% of the measured deposition volume.

  \item What are the main source areas of aeolian sediment at the Sand
    Motor mega nourishment?

    Aeolian sediment in the Sand Motor region originates from the dry
    beach area (aeolian zone, 33\%) and the intertidal and low-lying
    supratidal beach areas (mixed zones, 67\%). The relative
    importance of the mixed zones is notable as it is periodically
    flooded and the majority of the northern mixed zone is oriented
    unfavorable with respect to the wind.

  \end{enumerate}

  \bigskip

\item[Research objective B] Identify the main processes that govern
  aeolian sediment availability and supply at the Sand Motor mega
  nourishment (Chapter \ref{ch:smallscale}).

  \medskip

  The research questions and answers related to this objective are:

  \begin{enumerate}[{B}1]
  \item What processes govern the supply of aeolian sediment from the
    source areas?

    Aeolian sediment supply at the Sand Motor is governed by the
    development of a beach armor layer. The reduction of aeolian
    sediment supply from the dry beach area due to the formation of a
    beach armor layer increases the contribution of the intertidal
    beach areas to the total aeolian sediment supply. Aeolian sediment
    supply from the intertidal and low-lying supratidal beach areas
    (mixed zone) is governed by the periodic flooding of the
    intertidal beach. The timescales involved in the flooding and
    drying of the intertidal beach seems to be short, resulting in a
    swift response of the sediment supply to the instantaneous
    waterline position. Local deposits on the berm flat seem to act as
    temporary sediment source during high water or high soil moisture
    contents resulting in a continuous supply from the mixed zone.

  \item What processes govern the deposition of aeolian sediment in
    the deposition areas?

    Aeolian sediment deposits at the Sand Motor are found in areas
    with either limited wind shear, due to the presence of vegetation
    (dunes) or morphological feedback with the wind (lee of the Sand
    Motor crest), or high shear velocity thresholds, due to high soil
    moisture contents or free water surfaces (dune lake, lagoon and
    offshore). Local deposits on the berm flat seem to act as
    temporary sediment source during high water or high soil moisture
    contents resulting in a continuous supply from the intertidal
    beach. Alongshore variations in sediment deposition seem to be
    caused by blockage of aeolian sediment transport pathways.

  \item What bed surface characteristics are related to aeolian
    sediment supply?

    Significant changes in spatial gradients in aeolian sediment
    transport at the Sand Motor coincide with the presence of a beach
    armor layer. This suggests that beach armoring is a dominant
    process in the reduction of aeolian sediment
    supply. Spatial gradients in aeolian sediment transport
    also seem to be related to topographic features, like the
    transition from berm slope to berm flat. Such features seem to
    promote deposition of aeolian sediment (negative supply).
  \end{enumerate}

\bigskip

\item[Research objective C] Describe the generic influence of
  spatiotemporal variations in aeolian sediment availability on
  aeolian sediment transport in coastal environments (Chapter
  \ref{ch:model}).

  \medskip

  The research questions and answers related to this objective are:

  \begin{enumerate}[{C}1]
  \item What are existing approaches to describe the influence of
    aeolian sediment availability on aeolian sediment transport, what
    are the similarities and differences among them and which
    approaches are mutually exclusive?

    Three main approaches can be distinguished in literature: the
    shear velocity threshold, the critical fetch and the explicit
    formulation of sediment availability. All approaches are related,
    but differ in the amount of spatiotemporal variability in sediment
    availability they can allow. The approach based on critical fetch
    is mutually exclusive with the approach based on the explicit
    formulation of sediment availability as the latter provides the
    critical fetch as model result. The approach based on an explicit
    formulation of sediment availability is, in the form presented in
    this thesis, a spatiotemporal advection framework for the shear
    velocity threshold that in addition allows for feedback between
    aeolian sediment availability and transport as well as
    differentiation between the impact and fluid threshold.

  \item What processes that were identified to be relevant to aeolian
    sediment availability are not covered with sufficient accuracy by
    existing approaches?

    Spatiotemporal variations in beach armoring is shown to be a
    governing process at the Sand Motor mega nourishment. Especially
    the spatiotemporal variations are not sufficiently accurately
    described in existing models for aeolian sediment transport.

  \item What are the requirements for an approach that harmonizes
    existing, mutual inclusive approaches and is conceptually able to
    describe the processes relevant to aeolian sediment availability
    and transport?

    The approach is based on the legacy of aeolian research as
    relations between aeolian sediment availability and transport are
    abundantly available and should be used. The approach describes
    feedback between wind and transport as well as sediment
    availability and transport. It distinguishes between the fluid and
    impact threshold.
  \end{enumerate}

\bigskip

\item[Research objective D] Validate the numerical model approach to
  reproduce the location and size of sources for aeolian sediment at
  the Sand Motor mega nourishment (Chapter \ref{ch:hindcast}).

  \medskip

  The research questions and answers related to this objective are:

  \begin{enumerate}[{D}1]
  \item Can the calibrated numerical model reproduce the total aeolian
    sediment supply at the Sand Motor mega nourishment with any
    statistical significance?

    Yes. The total aeolian sediment supply over the course of 4 years
    is represented with an $\mathrm{R^2}$ value of 0.93 and an RMSE of
    $3 \cdot 10^4 ~ \mathrm{Mm^3}$.

  \item Can the calibrated numerical model reproduce the main source
    and deposition areas at the Sand Motor mega nourishment?

    Yes. The relative contribution of the intertidal beach was
    estimated to be 55\% based on the large scale sediment budget
    analysis, which is represented by the calibrated model.

  \item What implemented processes are significant to the model
    result?

    Both the drying of the intertidal beach and sediment sorting and
    beach armoring are crucial for the model result. Moreover, both
    the spatial and temporal variations affect the model result
    significantly.
  \end{enumerate}
\end{description}

%%% Local Variables:
%%% mode: latex
%%% TeX-master: "thesis"
%%% End:
